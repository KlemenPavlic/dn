% Začetek preambule
\documentclass[a4paper, 12pt]{article}
\usepackage[slovene]{babel}
\usepackage[utf8]{inputenc}
\usepackage[T1]{fontenc}
\usepackage{lmodern}
\usepackage{amsfonts,amsmath,amssymb}
\usepackage{graphicx}


% Moji ukazi, okolja,...
% oznake za števila
\newcommand{\N}{\mathbb{N}}
\newcommand{\Z}{\mathbb{Z}}
\newcommand{\Q}{\mathbb{Q}}
\newcommand{\R}{\mathbb{R}}
\newcommand{\C}{\mathbb{C}}
\newcommand{\F}{\mathbb{F}}
% opombe
\newenvironment{opomba}{\begin{flushleft} \textbf{Opomba}:}{\hfill \end{flushleft}}
% definicije
\newcounter{definitionCounter}
\addtocounter{definitionCounter}{1}
\newenvironment{definicija}{\begin{flushleft} \textit{\textbf{Definicija \arabic{definitionCounter}}}:}{\hfill \end{flushleft}\stepcounter{definitionCounter}}
% pojmi
\newcommand{\pojem}[1]{\textsc{#1}}
% dokazi
\newenvironment {dokaz}{\begin{flushleft} \textit{\textbf{Dokaz}}:}{\hfill $\square$\end{flushleft}}
% izreki
\newcounter{theoremCounter}
\addtocounter{theoremCounter}{1}
\newcounter{theoremCorollaryCounter}
\addtocounter{theoremCorollaryCounter}{0}
\newenvironment {izrek}{\begin{flushleft} \textsf{\textbf{IZREK \arabic{theoremCounter}}}:}{\hfill \end{flushleft}\stepcounter{theoremCounter}\stepcounter{theoremCorollaryCounter}\setcounter{corollaryCounter}{1}}
% leme
\newcounter{lemmaCounter}
\addtocounter{lemmaCounter}{1}
\newenvironment{lema}{\begin{flushleft} \textbf{Lema \arabic{lemmaCounter}}:}{\hfill \end{flushleft}\stepcounter{lemmaCounter}}
% posledice
\newcounter{corollaryCounter}
\addtocounter{corollaryCounter}{1}
\newenvironment  {posledica}{\begin{flushleft} \textsf{\textbf{Posledica \arabic{theoremCorollaryCounter}.\arabic{corollaryCounter}}}:}{\hfill \end{flushleft}\stepcounter{corollaryCounter}}
% dodatni ukazi
\usepackage{hyperref} % mora biti zadnji

% začetek dokumenta
\begin{document}
\thispagestyle{empty}
\noindent{\large
UNIVERZA V LJUBLJANI\\[1mm]
FAKULTETA ZA MATEMATIKO IN FIZIKO\\[5mm]
MATEMATIKA -- 2.~stopnja}
\vfill

\begin{center}{\large 
Klemen Pavlič\\[2mm]
{\bf Urejenostne algebrske strukture}\\[10mm]
Domača naloga}
\end{center}
\vfill

\noindent{\large
Velika Lašna, 2013}
\pagebreak
\newpage 

\begin{flushleft}
1. naloga
\end{flushleft}
Naj bo $G=\Z\times \Z$ z operacijo $(a,b)\circ (c,d) = (a+(-1)^b c, b+d)$.
\begin{enumerate}
\item[(a)] Dokažimo, da je $G$ grupa in da nima nobene linearne  ureditve.
\item[(b)] Dokažimo, da $G$ nima nobene mrežne ureditve.
\item[(c)] Konstruiraj na $G$ kako usmerjeno delno ureditev.
\end{enumerate}
\emph{Rešitev}
\begin{enumerate}
\item[(a)] Preverimo najprej, da je grupa. Jasno je, da je množica $\Z \times \Z$ zaprta za $\circ$. Preverimo asociativnost.
$$
\big[(a,b)\circ (c,d)\big] \circ (e,f) = (a+(-1)^b c, b+d) \circ (e,f) =
$$
$$
(a+(-1)^b c+ (-1)^{b+d}e, b+d+f),
$$
$$
(a,b)\circ \big[ (c,d) \circ (e,f) \big] = (a,b) \circ (c+(-1)^d e, d+f) =
$$
$$
(a+(-1)^b (c+(-1)^de), b+(d+f)) = (a+ (-1)^b c + (-1)^{b+d} e, b+d+f).
$$
Element $(0,0)$ je enota. Res, velja namreč
$$
(a,b)\circ (0,0) = (a+(-1)^b 0, b+0) = (a,b),
$$
$$
(0,0) \circ (a,b) = (0+(-1)^0 a, 0+b) = (a,b). 
$$
Inverz elementa $(a,b)$ je element $(a(-1)^{1-b},-b)$. Res, velja namreč
$$
(a,b) \circ (a(-1)^{1-b},-b) = (a+(-1)^b a (-1)^{1-b}, b-b) = (a-a,0 )= (0,0),
$$
$$
(a(-1)^{1-b}, -b) \circ (a,b) = (a(-1)^{1-b}+(-1)^{-b}a,-b+b)=
$$
$$
(a(-1)^{-b}(-1+1),0 ) = (0,0).
$$
Sledi, da je $(G,\circ)$ res grupa.

Pokažimo še, da $G$ nima nobene linearne ureditve. Denimo nasprotno, da na $G$ obstaja linearna ureditev. Potem za pozitiven stožec $P$ te ureditve velja $P\cup P^{-1} = G$. Preprost račun nam pokaže, da velja
$$
(e,f) \circ (a,b) \circ (e,f)^{-1} = (e+(-1)^f a + (-1)^{1+b} e,b).
$$
Vzemimo sedaj poljuben element oblike $(a,0), a\neq 0$. Njegov inverz je $(-a,0)$. Po drugi strani pa iz zgornje formule sledi, da je $(0,1)\circ(a,0)\circ (0,1)^{-1} = (-a,0)$. Vemo, da je bodisi $(a,0) \in P$ bodisi $(-a,0)\in P$. Predpostavimo lahko, da je $(a,0) \in P$, sicer ga zamenjamo z inverzom.  To pa pomeni, da je element $(-a,0) \in P\cap P^{-1} = \{e\} = \{(0,0)\}$. Sledi, da je $a=0$. Protislovje.

\item[(b)] Denimo nasprotno, da na $G$ obstaja mrežna ureditev. Naj bo spet $P$ njen pozitiven stožec. Če $P$ vsebuje element oblike $(a,0), a\neq 0$ potem kot v točki (a) pri dokazovanju neobstoja linearne ureditve pridemo do protislovja. Predpostavimo zato, da $P$ ne vsebuje elementov oblike $(a,0), a\neq 0$. Vzemimo sedaj element $(1,0)$ in označimo z $(a,b) = (1,0) \lor (0,0) \ge (1,0), (0,0)$. Velja torej $(a,b)\in P$ in zato je tudi $(e,f)\circ (a,b)\circ(e,f)^{-1} \ge 0$ za poljuben element $(e,f)$. Naj bo sedaj $(e,f) = (1,1)$. Opazujmo element
$$
(c,d) = (1,0)\circ (1,1)\circ (a,b) \circ (1,1)^{-1}.
$$
Ker je $(1,1)\circ (a,b) \circ (1,1)^{-1} \ge 0$, je $(c,d) \ge (1,0)$. Ker pa je $(a,b) \ge (1,0)$, je tudi $(1,1)\circ(a,b)\circ(1,1)^{-1} \ge (1,1)\circ (1,0) \circ (1,1)^{-1}$. S pomočjo te zveze izpeljemo oceno
$$
(c,d) = (1,0) \circ(1,1) \circ(a,b) \circ(1,1)^{-1} \ge (1,0) \circ (1,1) \circ(1,0) \circ(1,1)^{-1} =
$$
$$
(1,0) \circ (0,1) \circ(1,1)^{-1} = (1,1) \circ (1,1)^{-1} = (0,0).
$$
Velja torej, da je $(c,d) \ge (1,0), (0,0)$, torej je $(c,d)$ tudi njuna zgornja meja. Sledi, da je $(c,d) \ge (a,b)$. Preoblikujemo
$$
(a,b) \le (c,d)  =(1,0) \circ(1,1) \circ (a,b) \circ(1,1)^{-1} = 
$$
$$
(1,0) \circ (1,1) \circ (a,b) \circ (1,-1) = (1,0) \circ (1,1) \circ( a+(-1)^b,b-1) = 
$$
$$
(1,0) \circ (1,1) \circ(2a + (-1)^b, -1) \circ (a,b).
$$
Če pogledamo začetek in konec zadnje verige neenakosti, dobimo
$$
0 \le (1,0) \circ (1,1) \circ(2a + (-1)^b, -1) = (2-2a +(-1)^{b+1},0).
$$
Če je prva komponenta elementa na desni neničelna, pridemo v nasprotje s predpostavko, da $P$ ne vsebuje takih elementov. Veljati mora torej enakost, kar pomeni, da je $2-2a + (-1)^{b+1} = 0$. Preoblikujemo in dobimo, da mora biti $2+(-1)^{b+1} = 2a$. To pa ne gre, saj je na levi strani enačaja liho število, na desni strani enačaja pa sodo število. Sklenemo, da taki celi števili $a$ in $b$ ne obstajata. Prišli smo v nasprotje  s predpostavko, da na $G$ imamo mrežno ureditev.

\item[(c)] Definirajmo pozitiven stožec $P_{\ge} = \{(a,b); b > 0, a \text{ poljuben}\} \cup \{(0,0)\}$. Preverimo najprej, da je to res pozitiven stožec. Najprej preverimo, da velja $P_{\ge} \circ P_{\ge} \subseteq P_{ \ge}$. Naj bosta $(a,b), (c,d) \in P_{\ge}$. Če sta $b,d > 0$, potem je druga komponenta $(a,b)\circ (c,d)$ enaka $b+d > 0$ in je $(a,b)\circ (c,d) \in P_{\ge}$. Če je $d=0$, mora biti tudi $c=0$. Ker je $(0,0)$ .enota, je $(a,b)\circ (0,0) = (a,b)$ in ima drugo komponento pozitivno, torej je v pozitivnem stožcu $P_{\ge}$. Enako velja za primer, ko $b=0$ in $d>0$. Če pa sta $b=d=0$, mora biti še $a=c=0$ in je $(a,b)\circ(c,d) = (0,0) \in P_{\ge}$. Sledi $P_{\ge} \circ P_{\ge} \subseteq P_{\ge}$.

Pokažimo, da je $P_{\ge} \cap P_{\ge} = \{(0,0)\}$. Naj bo element $(a,b)$ v preseku. Potem mora veljati $(a,b) \ge 0$ in $(a,b)^{-1}=(a(-1)^{1-b},-b)\ge 0$. To je res natanko tedaj, ko je bodisi $b>0$ in $-b>0$ bodisi $a=b=0$ in $a(-1)^{1-b}=-b=0$. V prvem primeru pridemo do protislovja, torej mora veljati drugi primer, to pa pomeni, da je $a=b=0$ in da je v preseku res le enota.

Naj bo zdaj še $(c,d)\in G$ poljuben element in $(a,b)\in P_{\ge}$ poljuben element. Preverimo, da je $(c,d)\circ (a,b) \circ (c,d)^{-1} \in P_{\ge}$. Velja
$$
(c,d)\circ (a,b) \circ (c,d)^{-1} = (c,d) \circ (a,b) \circ (c(-1)^{1-d},-d) =
$$
$$
(c+(-1)^d a ,d+b) \circ (c(-1)^{1-d},-d)=
$$
$$
(c+(-1)^d a +(-1)^{b+d}c(-1)^{1-d},d+b-d)=
$$
$$
(c+(-1)^d a + (-1)^{1+b}c,b).
$$
Če je $b>0$ , potem je ta element iz $P_{\ge}$. Če pa je $b=0$, je tudi $a=0$ in zgornje naprej poenostavimo v $(c+(-1)^d a + (-1)^{1+b}c,b) = (c+(-1)c,0)=(0,0) \in P_{\ge}$. Sklenemo, da je $P_{\ge}$ res pozitiven stožec.

Pokažimo sedaj, da je $G$ v tej ureditvi usmerjena. Naj bosta torej $(a,b), (c,d)\in G$ poljubna elementa. Poiskati moramo njuno spodnjo in zgornjo mejo. Trdim, da je element $(0,\max\{b,d\} + 1)$ zgornja meja. Preverimo, da je $(a,b)\le (0,\max\{b,d\} + 1)$. To bo res natanko tedaj, ko bo $(0,\max\{b,d\} + 1)\circ (a,b)^{-1} = (0,\max\{b,d\} +1)\circ (a(-1)^{1-b},-b)=((-1)^{\max\{b,d\}+1} a (-1)^{1-b},\max\{b,d\} - b +1 ) \in P_{\ge}$. To pa velja, ker je druga komponenta $\max\{b,d\} - b +1 \ge 1 > 0$. Podobno vidimo, da je zgornji element tudi zgornja meja za $(c,d)$.

Trdim, da je element $(0,\min\{b,d\} - 1)$ spodnja meja za $(a,b)$ in $(c,d)$. Preverimo, da je $(0,\min\{b,d\} - 1) \le (a,b)$. Podoben račun kot zgoraj pokaže, da je to res natanko tedaj, ko je $(a,b)\circ (0,\min\{b,d\} -1)^{-1} = (a,b+1-\min\{b,d\}) \in P_{\ge}$. To pa spet velja, ker je druga komponenta $b+1-\min\{b,d\} \ge 1 > 0$. Enako se pokaže še $(0,\min\{b,d\} - 1) \le  (c,d)$. Sledi, da je $G$ usmerjena.
\end{enumerate}

\begin{flushleft}
2. naloga
\end{flushleft}
Naj bo $N$ podgrupa edinka grupe $G$ in naj bosta $N$ in $G/N$ delno urejeni. Dokažimo, da je množica 
$$
P:= N^+ \cup \{x\in G | N \neq xN \in (G/N)^+ \}
$$
pozitiven stožec neke delne ureditve na $G$ natanko tedaj, ko velja $gN^+g^{-1} \subseteq N^+$ za vse $g\in G$. V nadaljevanju predpostavimo, da je $P$ pozitiven stožec neke delne ureditve na $G$. Dokažimo:
\begin{enumerate}
\item[(a)] $(N,N^+)$ je urejenostno konveksna podgrupa v $(G,P)$.
\item[(b)] Če $g\in P\setminus N$, potem je $g > N$.
\item[(c)] Grupa $(G,P)$ je linearno urejena natanko tedaj, ko sta grupi $(N,N^+)$ in $(G/N, (G/N)^+)$ linearno urejeni.
\item[(d)] Če je $N\neq \{e\}$, potem je $(G,P)$ mrežno urejena grupa natanko tedaj, ko je $(N,N^+)$ mrežno urejena grupa in je $(G/N, (G/N)^+)$ linearno urejena grupa.
\item[(e)] Če je $(G,P)$ mrežno urejena grupa, potem je vložitev $N$ v $G$ pravilna (tj. ohranja tudi neskončne supremume in infimume).
\end{enumerate}
\emph{Rešitev}
\newline
Naj bo najprej $P$ pozitiven stožec neke ureditve na $G$ in $g\in G$ poljuben element. Ker je $N$ edinka je $gN^+ g^{-1} \subseteq N$ in ker je $P$ pozitiven stožec, je $gN^+ g^{-1} \subseteq N^+$.

Dokažimo še obrat. Naj za vsak $g\in G$ velja $g N^+ g^{-1} \subseteq N^+$. Dokažimo, da je $P$ pozitiven stožec. Pokažimo najprej, da je $P\cdot P \subseteq P$. Če $s,b\in N^+$, je $ab \in N^+$, ker je to pozitiven stožec. Če $x,y\in G$ taka, da je $N\neq xN \in (G/N)^+$ in $N \neq yN \in (G/N)^+$, je spet $(xy)N = (xN) (yN)\in (G/N)^+$, ker je to pozitiven stožec, ker pa je produkt strogo pozitivnih elementov strogo pozitiven, je še $(xy)N \neq N$. Če pa je $a\in N^+\subseteq N$ in $x\in G$ tak, da je $N\neq xN \in (G/N)^+$, potem je $(xa)N = (xN) (aN) = (xN)N = xN\in (G/N)^+ \setminus N$. Preverimo sedaj, da je $P\cap P^{-1} = \{e\}$. Velja $P^{-1} = N^- \cup \{ x\in G| N \neq xN \in (G/N)^-\}$. Jasno je $N^+ \cap N^- = \{e\}$ in ker velja tudi $(G/N)^+ \cap (G/N)^- = N$, sledi še $\{x\in G| N \neq xN \in (G/N)^+\} \cap \{x\in G| N \neq xN \in (G/N)^-\} = \emptyset$. Ker sta $N^+, N^- \subseteq  N$, velja še $N^+ \cap \{x\in G|  N \neq xN \in (G/N)^-\} = \emptyset$ in $N^- \cap \{x\in G|  N \neq xN \in (G/N)^+\} = \emptyset$. Sledi $P\cap P^{-1} = \{e\}$. Preveriti moramo še, da je za vsak $g\in G$ $gPg^{-1} \subseteq P$. Po predpostavki je $gN^+ g^{-1} \subseteq N^+$. Če pa je $x\in G$ tak, da je $N\neq xN \in (G/N)^+$, potem je $gxg^{-1} N = (gN) (xN) (gN)^{-1} \in (G/N)^+$, ker je to pozitiven stožec. Jasno je tudi, da je $(gN)(xN)(gN)^{-1} \neq N$, saj bi v nasprotnem primeru sledilo $xN = N$, kar pa ne velja. Sledi torej, da je tudi $gxg^{-1}$ tak, da je $N\neq (gxg^{-1})N \in (G/N)^+$. Zaključimo, da je $P$ res pozitiven stožec.
\begin{enumerate}
\item[(a)]Zadošča pokazati, da je za vsak $x\in N$ interval $[e,x]_G\subseteq N$. Naj bo torej $x\in N$, $y\in G$ in naj velja $e\le y\le x$. Ker je $y\ge e$, je $y\in P$. Če je $y\in N^+ \subseteq N$, smo končali. Predpostavimo torej lahko, da je $y\in \{x\in G| N \neq xN \in (G/N)^+\}$. Od tu sledi še, da je $y^{-1}N \neq N$ in da je $y^{-1}N\in (G/N)^- \setminus N$. Ker pa je $y\le x$, je $xy^{-1}\ge 0$ in zato $xy^{-1} \in P$. Če bi bil $xy^{-1}\in N$, bi bil v $N$ tudi element $x^{-1} (xy^{-1}) = y^{-1}$ in posledično $y\in N$, kar pa je v nasprotju z našo predpostavko. Sklenemo, da je tudi $xy^{-1} \in \{x\in G|N\neq xN \in (G/N)^+\}$. Konjugiramo z $y^{-1}$ in dobimo $y^{-1}xy^{-1}y = y^{-1} x \in P$. Enako kot prej pokažemo, da $y^{-1}x \notin N$. Sledi še, da je $y^{-1}x \in \{x\in G| N \neq xN \in (G/N)^+\}$. To pa pomeni, da je $(y^{-1}x)N \in (G/N)^+\setminus N$. Po drugi strani pa velja $(y^{-1}x)N = (y^{-1}N)(xN) = y^{-1}N\in (G/N)^- \setminus N$. Protislovje. Zaključimo, da je $y\in N$ in zato $N$ urejenostno konveksna.

\item[(b)] Naj bo $g\in P\setminus N$. Pokažimo, da je $g> N$. To je ekvivalentno temu, da za vsak $x\in N$ velja $g> x$. Denimo nasprotno, da obstaja $x\in N$, da je $g\le x$. Ker je $g\in P$, je $e \le g$. Velja torej $e \le g \le x$  oziroma $g\in [e,x]_G$  in ker je  $N$ urejenostno konveksna, to implicira, da je $g\in N$. Protislovje.

\item[(c)] Denimo najprej, da sta $N$ in $G/N$ linearno urejeni. Po predpostavki je $P$ pozitiven stožec. Pokažimo, da velja $P\cup P^{-1} = G$. Če je $g\in N$, potem zaradi linearne urejenosti slednje velja $g\in N^+ \cup N^-$. Če pa $g\notin N$, potem je $gN\in (G/N)\setminus N= \big((G/N)^+ \setminus N\big) \cup \big((G/N)^- \setminus N\big)$, ker je $G/N$ linearno urejena. Sledi, da za vsak $g\in G$ velja $g\in P\cup P^{-1}$ in zato je  $P\cup P^{-1} = G$, torej je $G$ linearno urejena.

Naj bo zdaj še $G$ linearno urejena. Pokažimo, da sta $N$ in $G/N$ linearno urejeni. Velja, da je $N\cap P$ pozitiven stožec na $N$. Res, velja namreč $(N\cap P)(N\cap P)\subseteq N\cap P$, ker $N$ podgrupa in $P$ pozitiven stožec. Prav tako je $(N\cap P) \cap (N\cap P)^{-1}  = (N\cap P ) \cap (N\cap P^{-1}) = N \cap P \cap P^{-1} = N\cap \{e\} = \{e\}$. Ker $N$ edinka in $P$ pozitiven stožec je tudi $g(N\cap P) g^{-1} \subseteq N\cap P$ za vsak $g\in G$. To je torej res pozitiven stožec. Ker pa je $P\cup P^{-1} = G$ in je $N\subseteq G$, je $(N\cap P) \cup (N\cap P^{-1})$ pokritje za $N$ in je zato $N$ linearno urejena.

Naj bo $\pi : G\rightarrow G/N$ urejenostni homomorfizem. Ker je $N$ po točki (a) urejenostno konveksna, je po izreku s predavanj množica $\pi(P)$ pozitiven stožec na $G/N$. Ker je $\pi$ homomorfizem, je $(\pi(P))^{-1} = \pi (P^{-1})$. Naj bo zdaj $N\neq gN \in G/N$ poljuben. Potem obstaja $h \in G\setminus N$, da je $\pi(h) = gN$. Ker je $G= P\cup P^{-1}$, je bodisi $h\in P\setminus N$ bodisi $h\in P^{-1} \setminus N$. V prvem primeru je po točki (b) element $h> N$ in ker je $\pi$ urejenotsni homomorfizem, je $\pi(h) = gN > N$, torej je $gN \in \pi(P)$. V drugem primeru pa je element $h^{-1} \in P\setminus N$ in je zato $h^{-1}> N$. Spet sledi, da je $(gN)^{-1} = (\pi(h))^{-1} = \pi(h^{-1}) > N$ in zato $gN \in (\pi(P))^{-1}$. Preverimo le še, da se ne more zgoditi, da bi obstajala $h_1,h_2\in G\setminus N$, ki bi zadoščala $\pi(h_i) = gN$ in $h_1 \in P, h_2 \in P^{-1}$. Recimo, da velja nasprotno. Ker je $h_1 \in P\setminus N$, je po (b) $h_1 > N$. Ker je $h_2 \in P^{-1}\setminus N$, je $h_2^{-1} > N$. Potem velja $h_1 h_2^{-1} > h_2^{-1} N^+ \ge N^+$ in zato je $h_1 h_2^{-1}> N^+$. Ker je $h_1h_2^{-1}\in P$, je še $h_1 h_2^{-1} > N^-$. Sledi, da je $h_1 h_2^{-1}>N$. Ker pa velja $\pi(h_1) = gN = \pi(h_2)$, mora biti $h_1 h_2^{-1}\in N$. Protislovje s tem, da je $h_1h_2^{-1}>N$. S tem smo pokazali, da je $\pi(P) \cup (\pi(P))^{-1} = G/N$ in je zato tudi $G/N$ linearno urejena.

\item[(d)] Naj bo $(N,N^+)$ mrežno urejena in $(G/N, (G/N)^+)$ linearno urejena. Pokažimo, da je $G$ mrežno urejana. Naj bosta $a,b\in G$. Poiskati moramo njun supremum. Naj bosta $a,b$ najprej v $N$. Ker je $N$ mrežno urejena, obstaja v $N$ supremum $a\lor_N b$. Pokažimo, da je to tudi supremum v $G$. Denimo nasprotno, da obstaja $c\in G$, da je $a,b \le c < a\lor_N b$. Ker je $N$ urejenostno konveksna, mora biti $c\in N$. To pa je v nasprotju s tem, da je $a\lor_N b$ supremum. Naj bo sedaj $b\in N$ in $a\notin N$. Potem je $aN \neq N$ in $a\in P\setminus N$ ali $a\in P^{-1}\setminus N$ (ker $G/N$ linearno urejena, je bodisi $aN > N$ bodisi $aN < N$). V prvem primeru je $a>N$, v drugem primeru pa $a<N$. Ker je $b\in N$, to pomeni, da je v prvem primeru $a>b$, v drugem primeru pa je $a<b$. Potem je v prvem primeru supremum $a$ in $b$ enak $a$, v drugem primeru pa $b$. Naj bosta zdaj še $a,b\notin N$. Potem velja $aN,bN \neq N$ in ker je $G/N$ linearno urejena velja bodisi $aN < bN$ bodisi $bN < aN$. Obravnavajmo le prvi primer, drugi gre namreč enako. Če je $aN < bN$, je tudi $aN  <b e = b $, ker $e\in N$ in od tod tudi $ae = a <b$.  Sledi, da je supremum $a$ in $b$ enak $b$. To je res supremum, saj ni manjše zgornje meje za element $b$.

Naj bo zdaj $(G,P)$ mrežno urejena. Pokažimo najprej, da je $(N,N^+)$ mrežno urejena. V ta namen pokažimo najprej, da je za vsak $a\in N$
 tudi $a\lor e\in N$. Če je $a$ pozitiven ali negativen, je to očitno res. Ker velja $a\lor e \ge e$, je $a\lor e \in P$. Če je $a\lor e \in N^+$ smo končali. Denimo nasprotno, da je $a\lor e \in \{x\in G| N\neq xN \in (G/N)^+\}$. Potem je po točki (b) $a\lor e > N$. Naj bo zdaj $h\in N$ poljuben element. Potem je $h(a\lor e) > hN$, kar pomeni, da je $h(a\lor e) >N$ in v posebej $h(a\lor e) > a,e$. To pomeni, da je $h(a\lor e)$ tudi zgornja meja za $a,e$. Veljati mora $h(a\lor e ) \ge a \lor e$. Ko pomnožimo prejšnjo neenakost z $(a\lor e)^{-1}$ dobimo $h\ge e$. Ker je bil $h\in N$ poljuben element, to pomeni, da so vsi elementi $N$ pozitivni. Za vsak pozitiven element pa velja, da je njegov inverz negativen in ker je $N$ podgrupa, mora ta inverz res biti v $N$. Sledi, da mora biti podgrupa $N$ kar trivialna. Protislovje s predpostavko, da je $N\neq \{e\}$. Zaključimo torej, da je za vsak element $a\in N$ element $a\lor e \in N$.

Naj bosta zdaj $a,b\in N$ poljubna elementa. Vemo, da v $G$ obstaja njun supremum. Pokažimo, da je tudi ta element podgrupe $N$. Denimo nasprotno, da $a\lor b\notin N$. Potem tudi element $ab^{-1} \lor e\notin N$, kar je v nasprotju z zgoraj ugotovljenim. Če bi namreč $ab^{-1} \lor e\in N$, potem je v $N$ tudi element $(ab^{-1} \lor e)b = a \lor b$, kar pa smo rekli da ne velja. To pomeni, da je za poljubna elementa $a,b\in N$ tudi $a\lor b\in N$ in zato je $N$ mrežna podgrupa.

Pokažimo še, da je $(G/N,(G/N)^+)$ linearno urejena podgrupa. Vemo že, da je podgrupa $N$ urejenostno konveksna mrežna podgrupa mrežne grupe $G$. Pokažimo, da za poljubna elementa $a,b\in G$ ki zadoščata $a\land b = e$ velja, da je vsaj eden izmed njiju iz podgrupe $N$. Denimo nasprotno, da $a,b\notin N$. Ker je $a\land b = e$, je $a,b\ge e$ in zato $a,b\in P$. Ker smo predpostavili, da $a,b\notin N$, je po točki (b) zagotovo res $a,b>N$. To pomeni tudi, da je $a,b>N^+$. Če bi v $N^+\setminus\{e\}$ obstajal kak element, potem bi zanj veljalo, da je spodnja meja elementov $a$ in $b$, ki je strogo večja od $a\land b =e$, kar je v nasprotju z definicijo infimuma. Ker na podgrupi $N$ nimamo trivialne ureditve (imamo mrežno ureditev), to pomeni, da je $N=\{e\}$. Protislovje s predpostavko. To pomeni, da je vsaj eden izmed elementov $a,b$ iz $N$. Pravkar dokazana lastnost pa je po izreku s predavanj ekvivalentna s tem, da je $G/N$ linearno urejena.

\item[(e)] Naj bo $L\subseteq N$ neskončna množica, za katero v $N$ obstaja supremum. Označimo s $c=\sup L$. Naj bo $\varphi : (N,N^+)\rightarrow (G,P)$ vložitev. Radi bi videli, da je $\sup \varphi(L) = \varphi(\sup L)= \varphi(c)$. Denimo, da temu ni tako. Potem obstaja $d\in G$, da je $\sup (\varphi(L)) =d < \varphi(c)$. Zaradi urejenostne konveksnoti mora biti potem tudi $d\in N$ (izberemo si nek element $x\in L\subseteq$, za katerega bo veljalo $\varphi(x)=x \le d < c = \varphi(c)$, zaradi urejenostne konveksnosti je $[x,c] \subset N$). Protislovje z definicijo supremuma.

\end{enumerate}

\begin{flushleft}
3. naloga
\end{flushleft}
Dokažimo, da ima vektorski prostor $\mathcal{C}^{(1)} [a,b]$ vseh zvezno odvedljivih funkcij na intervalu $[a,b]$ (z urejenostjo $f\ge g \Leftrightarrow f(x) \ge g(x)$ za vsak $x\in [a,b]$) Rieszovo interpolacijsko lastnost.
\newline
\emph{Opomba}
\newline
Rešitev te naloge je v članku "The Riesz Interpolation Property for the Space of Continously Differentiable Functions", ki ste nama ga dali na predavanjih. 
\newline
\emph{Rešitev}
\newline
Če ne zahtevamo zvezne odvedljivosti in opazujemo prostor zveznih funkcij se dokaz precej poenostavi. Prostor zveznih funkcij je namreč mreža v tej urejenosti, saj je za $f,g\in \mathcal{C}[a,b]$ tudi $f\lor g = \max\{f,g\} \in \mathcal{C}[a,b] $.

Dopolnimo še rešitev iz Lavričeve knjige, kjer pokaže, da ima $\mathcal{C}^1 [0,1]$ Rieszovo dekompozicijsko lastnost. Izberimo torej funkcije $g,f_1,f_2 \in \mathcal{C}^1[a,b]$ za katere velja $0\le g,f_1, f_2$ in $g\le f = f_1 + f_2$. Definiramo funkciji 
$$
g_i(x) = 
\left\{
\begin{array}{l l}
\frac{f_i(x) g(x)}{f(x)}, & \text{ če } f(x) \neq 0 \\
0, & \text{ če } f(x) = 0
\end{array}
\right.
, i = 1,2.
$$
Ker je $g\le f$ oziroma $\frac{g(x)}{f(x)} \le 1$ za $x$, kjer je $f(x) \neq 0$, takoj sledi, da je $0\le g_i  \le f_i$. Jasno je tudi, da velja $g= g_1 + g_2$. Preveriti moramo torej le še, da je $g_i \in \mathcal{C}^1[0,1]$. V točkah $x$ kjer je $f(x) \neq 0$ (torej je $f(x) > 0$), je funkcija $g_i$ očitno zvezno odvedljiva (sledi iz pravila za odvod kvocienta). Naj bo torej $x_0 \in (0,1)$ tak, da je $f(x_0) = 0$. Ker sta funkciji $f_i$ nenegativni, sledi $f_i(x_0) = 0, i=1,2$ in ker je $0\le g\le f$, je še $g(x_0) = 0$. Ker so funkcije $f_1,f_2,g$ nenegativne, to pomeni, da funkcije $f_1,f_2,g$ dosežejo minimum v točki $x_0$. Ker je ta iz notranjosti intervala, to pomeni, da je $f_1'(x_0) = f_2'(x_0)= g'(x_0)=0$. Potem pa za dovolj majhen $h\neq 0$ velja
$$
\big| \frac{g_i(x_0 + h) - g_i(x_0)}{h}\big| =\big|\frac{g_i(x_0+h)}{h}\big| \le \big|\frac{g(x_0+h)}{h}\big| = \big|\frac{g(x_0+h) - g(x_0)}{h}\big| \xrightarrow[h\rightarrow 0]{} 0.
$$
To pomeni, da je funkcija $g_i$ odvedljiva v $x_0$ in je $g'(x_0) = 0$. Če je $x$ tak, da je $f(x)>0$, potem iz neenakosti med fukcijami $f_1,f_2,g$ takoj lahko ocenimo
$$
|g_i'(x) - g_i'(x_0) | = |g_i'(x)| = \big|\frac{f_i'(x) g(x) + f_i(x) g'(x)}{f(x)} - \frac{f_i(x) g(x) f'(x)}{(f(x))^2}  \big| \le 
$$
$$
\big| f_i'(x)  \frac{g(x)}{f(x)} \big| + \big| \frac{f_i(x)}{f(x)} g'(x)\big| + \big| \frac{f_i(x)}{f(x)}\frac{g(x)}{f(x)} f'(x) \big| \le
$$
$$
|f_i'(x)| + |g'(x)| + |f'(x)| \xrightarrow[x\rightarrow x_0]{} |f_i'(x_0)| + |g'(x_0)| + |f'(x_0)| = 0.
$$
Sledi, da je $g_i'$ zvezna v $x_0$. Ostane nam še primer, ko je $f$ enaka 0 v krajišču. Poglejmo si najprej primer, ko je $x_0\in \{0,1\}$ tak, da je $f(x_0) = 0 $ in $f'(x_0) \neq 0$. Obravnavajmo primer, ko je $x_0 = 0$. Potem mora veljati še $f'(x_0) > 0$ (funkcija ne more biti padajoča. ker je nenegativna). Enako kot zgoraj sledi še $f_1(0) = f_2(0) = g(0) = 0$. Ker je $f'(0) \neq 0$, je še $f(h) > 0$ za dovolj majhen pozitiven $h$. Za take $h$ potem velja formula
$$
g_i'(h) = \frac{f_i'(h) g(h) + f_i(h) g'(h)}{f(h)} - \frac{f_i(h)g(h)f'(h)}{(f(h))^2}. 
$$
Velja
$$
\frac{g_i(0+h)-g_i(0) }{h} = \frac{g_i(h)}{h} = \frac{1}{h} \frac{f_i(h) g(h)}{f(h)}\frac{h}{h} = \frac{(f_i(h)/h)(g(h)/h)}{f(h)/h}=
$$
$$
\frac{((f_i(h)-f_i(0))/h)((g(h)-g(0))/h)}{(f(h)-f(0))/h}\xrightarrow[h\rightarrow 0]{} \frac{f_i'(0) g'(0)}{f'(0)}.
$$
Sledi torej, da je $g_i'(0) = \frac{f_i'(0) g'(0)}{f'(0)}$. Sledi odvedljivost. Preverimo še zveznost odvoda. Preoblikujemo zgoraj zapisano formulo za odvod kvocienta pri majhnih pozitivnih $h$. Iz Taylorjevega razvoja sledi, da je $f_i(h) = 0 + f_i'(0) h + o(h)$ in $g(h) = 0 + g'(0)h+o(h)$. Zapišemo lahko 
$$
\frac{f_i'(h) g(h) + f_(h) g'(h)}{f(h)} = \frac{f_i'(h) (g'(0) + \frac{o(h)}{h})  +  (f_i'(0) + \frac{o(h)}{h} )g'(h)}{f(h) / h}\xrightarrow[h\rightarrow 0]{} \frac{ 2f_i'(0)g'(0)}{f'(0)}
$$
in
$$
\frac{f_i(h)g(h) f'(h)}{(f(h))^2} = \frac{(f_i'(0) + \frac{o(h)}{h}) (g'(0) + \frac{o(h)}{h})f'(h) }{(f(h)/h)^2} \xrightarrow[h\rightarrow 0]{} \frac{f_i'(0)g'(0)}{f'(0)}.
$$
Iz zgornje enačbe za odvod kvocienta za majhne pozitivne $h$ potem sledi, da je $\lim_{h\rightarrow 0}g_i'(h) = \frac{ 2f_i'(0)g'(0)}{f'(0)} - \frac{f_i'(0)g'(0)}{f'(0)} = \frac{f_i'(0)g'(0)}{f'(0)} = g_i'(0)$ (zadnja enakost velja po zgoraj ugotovljenem). To pa pomeni, da je $g_i'$ zvezen v 0. Za krajišče 1 velja podoben premislek ali pa apliciramo pravkar ugotovljeno na funkcije $g(1-x), f_i(1-x)$.

Obravnavamo še $x_0\in \{0,1\} $ in $f(x_0) = f'(x_0) = 0$. Spet pogledamo najprej primer, ko je $x_0 = 0$ (za krajišče 1 podobno). Spet sledi enako kot zgoraj, da je $f_1(0) = f_2(0) = g(0) = 0$. Za majhne pozitivne $h$ potem spet velja
$$
0 \le \frac{g_i(h) - g_i(0)}{h} = \frac{g_i(h)}{h}\le \frac{f_i(h)}{h} = \frac{f_i(h) - f_i(0)}{h} \xrightarrow[h\rightarrow 0]{} f_i'(0) = 0.
$$
Sledi odvedljivost v 0 in $g_i'(0) = 0$. 
Enako kot zgoraj izpeljemo oceno
$$
|g_i'(h) - g_i'(0)| = |g_i'(h)| \le |f_i'(h) | + |g'(h)| + |f'(h)| \xrightarrow[h\rightarrow 0]{} 0,
$$
od koder sledi še zveznost odvoda v točki 0. Sledi, da je v vseh primerih $g_i \in \mathcal{C}^1[a,b]$ in zato dani prostor res ima Rieszovo dekompozicijsko lastnost.




\begin{flushleft}
4. naloga
\end{flushleft}
Naj bo $I$ neskončna množica.
\begin{enumerate}
\item[(a)] Dokažimo, da je $\oplus_{i\in I} \Z$ strnjena podgrupa v $\prod_{i\in I} \Z$.
\item[(b)] Dokažimo, da faktorska grupa $\prod_{i\in I} \Z / \oplus_{i\in I} Z$ ni arhimedska.
\end{enumerate}
\emph{Rešitev}
\begin{enumerate}
\item[(a)] Elementi $\oplus_{i\in I}$ imajo le končno mnogo neničelnih komponent. Jasno je, da je $\oplus_{i \in I} \Z$ podgrupa, saj je vsota dveh elementov, ki imata le končno neničelnih komponent, tudi sama take oblike. Isto velja za inverze. Pokažimo raje, da je strnjena. Naj bo torej $a = (a_i)_{i\in I}\in \oplus_{i \in I} \Z$ in $b = (b_i)_{i\in I}\in \prod_{i\in I} \Z$. Naj zanju velja še $|b| \le |a|$. To pomeni, da za vsak $i\in I$ velja $|b_i| \le |a_i|$. Naj bo sedaj $A= \{ i \in I; a_i \neq 0\}$ in $A^C = \{i\in I; a_i =0\}$. Po predpostavki je $|A| < \infty$. Iz zveze $|b_i| \le |a_i| $ pa sledi še, da je za vsak $i\in A^C$ element $b_i=0$. Sledi, da ima $b$ lahko neničelne komponente le na mestih, na katerih so indeksi iz množice $A$. To pa pomeni, da ima tudi $b$ lahko največ končno mnogo neničelnih komponent in je zato tudi $b\in \oplus_{i\in I} \Z$. Zaključimo, da je $\oplus_{i\in I}\Z$ strnjena podgrupa v $\prod_{i \in I} \Z$.
\item[(b)] Premislimo najprej, kdaj v faktorski grupi $\prod_{i\in I} \Z/ \oplus_{i\in I} \Z$ enačimo dva elementa. Po definiciji je $a + \oplus_{i\in I} \Z = b + \oplus_{i\in I} \Z $ natanko tedaj, ko je $a-b\in \oplus_{i\in I} \Z$. To pa velja natanko tedaj, ko ima $a-b$ le končno mnogo neničelnih komponent oziroma ko se $a$ in $b$ razlikujeta le na končno mnogo komponentah. Od tod takoj sledi, da so v ekvivalenčnem razredu enote natanko vsi tisti elementi, ki imajo le končno neničenih komponent. Prav tako velja še, da je $a\le b$ v faktorski grupi natanko tedaj, ko ima $b-a$ le končno negativnih komponent. Vzemimo sedaj elementa $a=(1,1,1,\dots)$ in $b=(1,2,3,\dots)$. Potem je $b-na = (1-n,2-n,3-n,\dots, -2,-1,0,1,2,3,\dots)$. Sledi, da ima ta  razlika za vsak $n$ le končno mnogo negativnih komponent in je zato  pozitiven element. To je ekvivalentno temu, da je za vsak $n\in \N$ veljavna zveza $na \le b$, vendar pa je $a\neq 0$, saj se razlikueta v vseh komponentah. Sledi, da faktorska grupa $\prod_{i\in I}\Z/ \oplus_{i\in I} \Z$ res ni arhimedska.
\end{enumerate}

\begin{flushleft}
5. naloga
\end{flushleft}
Naj bo $G$ aditivna  grupa vseh celoštevilskih $2\times 2$ matrik z delno urejenostjo
$$
\begin{bmatrix}
a_1 & b_1 \\
c_1 & d_1 \\
\end{bmatrix}
\le 
\begin{bmatrix}
a_2 & b_2 \\
c_2 & d_2 \\
\end{bmatrix}
\Leftrightarrow
\begin{array}{l}
a_1 < a_2 \text{ ali}\\
a_1 = a_2 \text{ in } b_1 < b_2 \text{ ali}\\
a_1 = a_2 \text{ in } b_1 = b_2 \text{ in } c_1 \le c_2 \text{ in } d_1 \le d_2\\
\end{array}.
$$
\begin{enumerate}
\item[(a)] Dokaži, da je $G$ mrežno urejena.
\item[(b)] Izračunaj vrednosti koordinatnih matrik $E_{ij}$.
\item[(c)] Dokaži, da smo v (b) dobili vse prapodgrupe  v $G$.
\item[(d)] Opiši Hahnovo vložitev za grupo $G$.
\end{enumerate}
\emph{Rešitev}
\begin{enumerate}
\item[(a)]Trdim, da je množica
$$
P=\big\{
\begin{bmatrix}
a_1 & b_1 \\
c_1 & d_1 \\
\end{bmatrix}
;
\begin{array}{l}
a_1 >0 \text{ ali}\\
a_1 = 0 \text{ in } b_1 > 0 \text{ ali}\\
a_1 = 0 \text{ in } b_1 = 0 \text{ in } c_1 \ge 0 \text{ in } d_1 \ge 0\\
\end{array}
\big\}
$$
pozitiven stožec. Če ločimo nekaj primerov, z lahkoto pokažemo, da je $P+P \subseteq P$. Pokažimo še, da je $P\cap -P = \{0\}$. Če je 
$\begin{bmatrix}
a_1 & b_1 \\
c_1 & d_1\\
\end{bmatrix}
$
iz $P\cap -P$, potem mora veljati $a_1 > 0$ ali $(a_1 = 0 \text{ in } b_1 > 0)$ ali $(a_1=0 \text{ in } b_1 = 0 \text{ in } c_1 \ge 0 \text{ in } d_1 \ge 0)$ in hkrati še  $-a_1 > 0$ ali $(-a_1 = 0 \text{ in } -b_1 > 0)$ ali $(-a_1=0 \text{ in } -b_1 = 0 \text{ in } -c_1 \ge 0 \text{ in } -d_1 \ge 0)$. Sledi, da mora veljati $a_1 =b_1 =0$ in $0\le c_1 \le 0, 0 \le d_1 \le 0$, torej še $c_1 = d_1 = 0$. Sledi, da je v preseku res samo enota in ker je ta grupa komutativna, je ta množica res pozitiven stožec. 
Pokažimo še, da je mreža v tej urejenosti. Naj bosta 
$
A=
\begin{bmatrix}
a_1 & b_1 \\
c_1 & d_1 \\
\end{bmatrix}
$
in
$
B = 
\begin{bmatrix}
a_2 & b_2 \\
c_2 & d_2 \\
\end{bmatrix}
$
poljubni matriki. Poiščimo njun supremum. Če je $a_1 > a_2 $, potem je $A \lor B = A$ in podobno  je v primeru  $a_1 < a_2$ $A\lor B = B$. Naj bo torej zdaj $a_1 = a_2$. Če je še $ b_1 < b_2 $ je $A\lor B = B$, če pa je $b_1 > b_2$, pa je $A \lor B = A$. Opazujmo zdaj primer, ko je $a_1 = a_2 $ in $b_1 = b_2$. Ločimo še nekaj primerov. Če je 
\begin{itemize}
\item $c_1 \ge c_2 $ in $d_1\ge d_2$ je $A\lor B = A$,
\item $c_1 \le c_2$ in $d_1 \le d_2$ je $A\lor B = B$,
\item $c_1 \ge c_2$ in $d_2 \ge d_1$ je  $A\lor B=
\begin{bmatrix}
a_1 & b_1\\
c_1 & d_2 \\
\end{bmatrix}$,
\item $c_2 \ge c_1$ in $d_1 \ge d_2$ je  $A\lor B=
\begin{bmatrix}
a_1 & b_1\\
c_2 & d_1 \\
\end{bmatrix}$.
\end{itemize}
Vsi ti elementi so res pravi supremumi. Podobno poiščemo še infimume. Sledi, da je $G$ v dani urejenosti res mreža.
\item[(b)] Določimo sedaj vrednosti koordinatnih matrik $E_{ij}$. Vemo, da je vrednost maksimalna strnjena podgrupa, ki še ne vsebuje danega elementa. Začnimo z elementom $E_{11}$. Trdim, da je 
$$
V_{E_{11}} = 	\big\{ 
\begin{bmatrix}
0 & b_1 \\
c_1 & d_1 \\
\end{bmatrix}; b_1,c_1,d_1 \in \Z
  \big\}.
$$
Očitno je $V_{E_{11}}$ podgrupa in $E_{11}\notin V_{E_{11}}$. Preverimo, da je $V_{E_{11}}$ strnjena. Denimo, da za matriki 
$B=\begin{bmatrix}
a_2 & b_2\\
c_2 & d_2\\
\end{bmatrix}
\in G
$
in
$
A=\begin{bmatrix}
0 & b_1\\
c_1 & d_1\\
\end{bmatrix}
\in V_{E_{11}}
$
velja $|B| \le |A|$. To je ekvivalentno temu, da je $|a_2| < 0 $ ali $|a_2| = 0$ in $|b_2| < |b_1|$ ali $|a_2| = 0$ in $|b_2| = |b_1| $ in $|c_2| \le |c_1|$ in $|d_2| \le |d_1|$. Jasno je, da prvi pogoj ne more biti izpolnjen, kar pomeni, da mora biti $a_2 = 0$ in je zato $B\in V_{E_{11}}$. Sledi, da je $V_{E_{11}}$ strnjena. Ta premislek nam tudi pokaže, da je $V_{E_{11}}$ definirana kot zgoraj res maksimalna strnjena podgrupa, ki ne vsebuje $E_{11}$. Če bi namreč na mestu $(1,1)$ vsebovala nek neničelen element, bi morala ta podgrupa zaradi strnjenosti in  definicije urejenosti vsebovati tudi $E_{11}$.

Trdim, da je 
$$
V_{E_{12}} = \big\{
\begin{bmatrix}
0 & 0 \\
c_1 & d_1 \\
\end{bmatrix};
c_1 , d_1 \in \Z
\big\}.
$$
Jasno je to podgrupa, ki ne vsebuje $E_{12}$. Denimo, da za neko matriko $B$ oblike kot zgoraj velja, da je $|B| \le |A|, A \in V_{E_{12}}$. To je ekvivalentno temu, da je $|a_2| < 0$ ali $|a_2| = 0 $ in $|b_2| < 0$ ali $|a_2| = 0 $ in $|b_2| = 0 $ in $|c_2| \le |c_1|$ in $|d_2| \le |d_1|$. Spet vidimo, da je to lahko res le v tretjem primeru, torej je $a_2 = b_2 =0$ in je zato $B\in V_{E_{12}}$. Podgrupa je torej strnjena. Premislimo še, da je maksimalna strnjena podgrupa, ki ne vsebuje elementa $E_{12}$. Če bi vrednost elementa $E_{12}$ vsebovala nek element, ki ima na mestu $(1,1)$ nekaj neničelnega, bi zaradi strnjenosti vsebovala element $E_{12}$ (prva možnost iz definicije urejenosti), to pa ne gre. Podobno bi v primeru, da bi na mestu $(1,2)$ vsebovala neničelen element zaradi strnjenosti po drugi možnosti iz definicije urejenosti vsebovala element $E_{12}$, kar nas spet privede v protislovje z definicijo vrednosti. Sklenemo, da je zgoraj definirana podgrupa res vrednost elementa $E_{12}$.

Vrednost elementa $E_{21}$ je 
$$
V_{E_{21}} = \big\{
\begin{bmatrix}
0 & 0 \\
0 & d_1 \\
\end{bmatrix};
d_1 \in \Z
\big\}.
$$
To je res podgrupa, ki ne vsebuje elementa $E_{21}$. Preverimo, da je strnjena. Naj bo $B$ tak kot zgoraj in $|B| \le |A|, A\in V_{E_{21}}$. To pomeni da je $|a_2| < 0$ ali $ |a_2| =0 $ in $|b_2|<0$ ali $|a_2| = 0 $ in $|b_2| = 0$ in $|c_2|\le 0$ in $|d_2| \le | d_1|$. To implicira, da je $a_2 = b_ 2 = c_2 = 0$ in je zato $B\in V_{E_{21}}$, torej je to strnjena podgrupa. Premislimo še, da je maksimalna strnjena podgrupa, ki ne vsebuje elementa $E_{21}$. Podobno kot zgoraj, bi v primeru, da bi vrednost elementa $E_{21}$ vsebovala matriko, ki bi imela nekaj neničelnega na mestu $(1,1)$ ali $(1,2)$ zaradi strnjenosti takoj sledilo, da vsebuje še $E_{21}$, kar pa ne gre. Seveda tudi v primeru, ko bi na mestu $(2,1)$ vsebovala nekaj neničelnega (v prvi vrstici pa le dve ničli) po tretji možnosti iz definicije urejenosti zaradi strnjenosti podgrupe sledi, da vsebuje $E_{21}$, kar pa seveda ne gre. Našli smo torej vrednost elementa $E_{21}$.

Vrednost elementa $E_{22}$ pa je 
$$
V_{E_{22}} = \big\{
\begin{bmatrix}
0 & 0 \\
c_1 & 0 \\
\end{bmatrix};
c_1 \in \Z
\big\}. 
$$
To je res podgrupa, ki ne vsebuje $E_{22}$. Enako kot v prejšnjem primeru vidimo, da je strnjena. Na enak način kot v prejšnjem primeru vidimo še, da je to res maksimalna strnjena podgrupa, ki ne vsebuje elementa $E_{22}$, torej je to res vrednost $E_{22}$.

\item[(c)] Dokazati moramo, da smo v prejšnji točki dobili vse prapodgrupe grupe $G$. To preprosto sledi iz zgornjih premislekov o maksimalnosti strnjene podgrupe, ki ne vsebuje določenega elementa. Če imamo podgrupo, ki ima na mestu $(1,1)$ nekaj neničelnega, so zaradi strnjenosti in definicije urejenosti v njej tudi vse štiri matrične enote in dobimo kar celo grupo $G$. Denimo torej, da imamo podgrupo, ki vsebuje le matrike, ki imajo 0 na mestu $(1,1)$, drugje pa poljuben element. Tako dobimo $V_{E_{11}}$ (zaradi strnjenosti to podgrupo dobimo vedno, ko podgrupa vsebuje matriko, ki ima na mestu $(1,2)$ nekaj neničelnega) . Denimo sedaj, da imamo na mestih $(1,1),(1,2)$ v vseh matrikah iz podgrupe same ničle. Če imamo v podgrupi kako matriko, ki ima na mestu $(2,1)$ nekaj neničelnega in še neko matriko, ki ima na mestu $(2,2)$ neničelen element, potem zaradi strnjenosti dobimo podgrupo $V_{E_{12}}$. Če podgrupa vsebuje le matrike, ki imajo neničelne elemente le na mestu $(2,1)$ (oziroma $(2,2)$), potem zaradi strnjenosti vsebuje element $E_{21}$ (oziroma $E_{22}$) in je zato enaka $V_{E_{22}}$ (oziroma $V_{E_{21}}$). Če podgrupa ne vsebuje nobene matrike, ki bi imela na mestu $(2,1)$ ali $(2,2)$ nekaj neničelnega, potem ta podgrupa ne more vsebovati nobene matrične enote in je zato to trivialna podgrupa. 

\item[(d)] Po (a) vemo, da je $G$ mrežno urejena. Po točkah (b) in (c) vemo, da so vse vrednosti elementov iz $G$ ravno $\Lambda=\{V_{E_{11}}, V_{E_{12}}, V_{E_{21}}, V_{E_{22}}\}$. Množico $\Lambda$ uredimo z relacijo nasprotno inkluziji, kar pomeni, da je za $A,B\in \Lambda$ $A\le B$ natanko tedaj, ko je $B\subseteq A$. Dobimo neenakosti $V_{E_{11}} \le V_{E_{12}} \le V_{E_{21}}, V_{E_{22}}$, pri čemer zadnja dva elementa nista primerljiva.  Grupo $G$ potem vložimo v Hahnov produkt, to je v delno urejeno grupo $\prod_{\lambda \in \Lambda} \R_{\lambda} = \R_{V_{E_{11}}} \times \R_{V_{E_{12}}} \times\R_{V_{E_{21}}} \times \R_{V_{E_{22}}}$  z operacijo seštevanja. 
\end{enumerate}

\begin{flushleft}
6. naloga
\end{flushleft}
Naj bo $(F,<)$ urejen obseg opremljen z intervalsko topologijo (to je topologija, ki je generirana z odprtimi intervali s krajišči v $F$). Dokažimo, da so naslednje preslikave zvezne: vsota, produkt (kot preslikavi iz $F\times F$ v $F$) in inverz (kot preslikava iz $F \setminus \{0\}$ v $F\setminus \{0\}$).
\newline
\emph{Rešitev}
\newline
Začnimo z inverzom. Označimo z $f(x) = x^{-1}$. Naj bo $x\in F\setminus\{0\}$ in $x^{-1}$ njegov inverz. Naj bo $(a,b)$ okolica $x^{-1} = f(x)$. Ker ima vsak element enolično določen inverz, obstajata $a_0, b_0$, da je $f(a_0) = a = a_0^{-1}$ in $f(b_0) = b = b_0^{-1}$. Ker je $a< x <b$ in ker invertiranej obrne urejenost, je $x\in (b_0,a_0)$ in $f((b_0,a_0)) = (a_0^{-1}, b_0^{-1}) = (a,b)$. Sledi, da je $f$ zvezna v $x$ in ker je bil $x$ poljuben, je invertiranje zvezna preslikava.

DOPOLNITI

\begin{flushleft}
7. naloga
\end{flushleft}
Naj bo $R$ množica vseh realnih števil, ki so ničle polinomov s celoštevilskimi koeficienti.
\begin{enumerate}
\item[(a)] Dokažimo, da ima $R$ števno mnogo elementov.
\item[(b)] Dokažimo, da je $R$ zaprt za elementarne računske operacije.
\item[(c)] Dokaži, da je urejeni obseg $R$ realno zaprt.
\item[(d)] Dokaži, da interval $[0,1]$ v $R$ ni kompakten v intervalski topologiji $R$ (tj. v topologiji podedovani iz $\R$).
\end{enumerate}
\emph{Rešitev}
\begin{enumerate}
\item[(a)] Vemo, da je število realnih ničel polinoma manjše ali enako njegovi stopnji. To pa pomeni, da je vseh realnih števil, ki so ničle polinomov s celoštevilskimi koeficienti kvečjemu toliko, kot je polinomov s celoštevilskimi koeficienti. Velja $|R| \le |\cup_{i=1}^\infty \mathbb{P}_n(\Z)|$, kjer je $\mathbb{P}_n(\Z)$ množica vseh polinomov s celoštevilskimi koeficienti stopnje kvečjemu $n$. Velja $|\mathbb{P}_n(\Z)| = |\Z^{n+1}| = |\N|$. Sledi, da ima  $R$ največ števno elementov. Jasno pa je tudi, da je $\Z \subseteq R$, saj je vsak $a\in \Z$ ničla moničnega celoštevilskega polinoma $x-a$. Sledi, da ima $R$ števno elementov.

\item[(b)] Preverimo najprej, da je $R$ zaprta za invertiranje. Naj bo $0\neq a\in R$ in $p\in \Z[x]$ polinom oblike $p(x) = \alpha_0 x^n + \alpha_1 x^{n-1} + \dots  + \alpha_{n-1}x +\alpha_n$, ki zadošča $0=p(a) =  \alpha_0 a^n + \alpha_1 a^{n-1} + \dots  + \alpha_{n-1}a +\alpha_n = a^n (\alpha_0 + \alpha_1 a^{-1} + \dots + \alpha_{n-1} (a^{-1})^{n-1} + \alpha_n (a^{-1})^{n})$. Sledi, da je $(\alpha_0 + \alpha_1 a^{-1} + \dots + \alpha_{n-1} (a^{-1})^{n-1} + \alpha_n (a^{-1})^{n}) = 0$ in zato za polinom $\tilde{p}(x) = (\alpha_0 + \alpha_1 x + \dots + \alpha_{n-1} x^{n-1} + \alpha_n x^{n})$ velja $\tilde{p}(a^{-1}) = 0$. Sledi, da je $a^{-1}\in R$.

Preverimo sedaj, da je $R$ zaprta za seštevanje. Najprej pa opazimo, da namesto ničel polinomov s celimi koeficienti lahko opazujemo ničle polinomov z racionalnimi koeficienti, saj je vsaka ničla polinoma z racionalnimi koeficienti tudi ničla polinoma s celimi koeficienti, ki ga dobimo iz prvega tako, da ga pomnožimo z najmanjšim skupnim večkratnikom imenovalcev koeficientov. Naj bo torej $a$ ničla polinoma $p(x) \in \Q[x]$ in $b$ ničla polinoma $q(x)\in \Q[x]$. Če sta $p(x) = \alpha_0 x^n + \alpha_1 x^{n-1} + \dots + \alpha_{n-1}x + \alpha_n$ in $q(x) = \beta_0 x^m + \beta_1 x^{m-1} + \dots + \beta_{m-1}x + \beta_m$ izbrana polinoma, potem velja $ \alpha_0 a^n + \alpha_1 a^{n-1} + \dots + \alpha_{n-1}a + \alpha_n = 0$  in $\beta_0 b^m + \beta_1 b^{m-1} + \dots + \beta_{m-1}b + \beta_m=0$. To pomeni, da sta množici elementov $\{1,a,\dots, a^n\}$ in $\{1,b,\dots,b^m\}$ zaporedoma linearno odvisni v $\Q$ vektorskih prostorih $A$, ki ga napenjajo vektorji $\{a^i\}_{i=0}^n$ in $B$, ki ga napenjajo vektorji $\{b^j\}_{j=0}^m$. Velja $\dim_{\Q} A < n+1$ in $\dim_{ \Q} B < m+1$. Oglejmo si sedaj $Q$ vektorski prostor $C$, ki ga generirajo vektorji $\{(a+b)^i\}_{i=0}^r$. Če razvijemo element $(a+b)^i$ po potencah $a$ in $b$ in upoštevamo linearno odvisnost potenc $a$ in $b$, z lahkoto zapišemo $(a+b)^i$ kot linearno kombinacijo elementov $a^i b^j, 0\le i \le n, 0 \le j \le m$. Sledi $\dim_{\Q}C < (n+1)(m+1)$. Če torej v $C$ izberemo $r=(n+1)(m+1)$ različnih vektorjev $(a+b)^i$, so le ti $\Q$ linearno odvisni med sabo, kar pomeni, da obstaja polinom iz $\Z[x]$, ki uniči $a+b$.

Preverimo še, da je $R$ zaprta za produkt (zastonj dobimo še zaprtost za aditivne inverze, saj $R$ vsebuje -1, ki je ničla polinoma $x+1$). Postopamo podobno kot pri vsoti. Naj bodo $a,b,p,q,A,B$ kot zgoraj. Naj bo $D$ vektorski prostor nad $\Q$, ki ga napenjajo vektorji $\{(ab)^i\}_{i=1}^r$. Upoštevajoč linearno odvisnost elementov $a^i, b^j, 0\le i \le n, 0\le j \le m$, spet lahko vsak vektor oblike $(ab)^k$ zapišemo kot vsoto elementov oblike $a^ib^j, 0\le i  \le n, 0\le j \le m$. Sledi, da je $D$ podprostor $C$ im zato je $\dim_{\Q}D \le \dim_{\Q}C < (n+1)(m+1)$. Če v $D$ zopet izberemo $r=(n+1)(m+1)$ različnih elementov $\{(ab)^i\}_{i=0}^r$, bodo torej linearno odvisni med sabo in zato zopet obstaja polinom v $\Z[x]$, ki uniči $ab$.

Sledi, da je $R$ polje.

\item[(c)] Polje $R$ podeduje ureditev realnih števil, v kateri je $R$ linearno urejeno polje. Po trditvi s predavanj bo $R$ realno zaprt  natanko tedaj, ko bo $R[i]$ algebraično zaprt obseg. To pa velja, saj je $R[i]$ ravno polje vseh algebraičnih števil, ki pa je algebraično zaprto.
\item[(d)] Vemo, da je število $\pi$ transcendentno, torej tudi $\pi^{-1} \notin R$. Po drugi strani je $\pi^{-1} \in [0,1]$. Velja še, da zaporedje $\frac{1}{3}, \frac{1}{3.1}, \frac{1}{3.14}, \frac{1}{3.141},\dots$ konvergira k $\pi^{-1}$ in za vsak člen zaporedja z lahkoto najdemo polinom, katerega ničla je ta člen. Res, za člen $a_n$ obstaja monični polinom z racionalnimi koeficienti $x-a_n$, ki ga z lahkoto zapišemo kot polinom s celoštevilskimi koeficienti. To naredimo tako, da $a_n$ zapišmo kot ulomek in pomnožimo polinom z imenovalcem. Našli smo torej zaporedje elementov iz $R\cap[0,1]$, ki konvergirajo k elementu, ki ni v $R\cap[0,1]$. Sledi, da dana množica ni kompaktna.
\end{enumerate}

\begin{flushleft}
8. naloga
\end{flushleft}
Naj bo $Q$ podmnožica v $\R(X)$, ki vsebuje 0 in katere neničelni elementi so oblike $X^m g(X)$, kjer je $m\in \Z$ in je $g(X) \in \R(X)$ definiran in strogo pozitiven v 0.
\begin{enumerate}
\item[(a)] Dokažimo, da je množica $Q$ ureditev na $\R(X)$.
\item[(b)] Dokažimo, da sta za vsak $a\in \R$ množici $Q_{a^+} = \{r\in \R(X)| r(X-a) \in Q\}$ in $Q_{a^-} = \{ r \in \R(X)| r(a-X) \in Q\}$ ureditvi obsega $\R(X)$. Dokažimo, da sta ureditvi tudi množici $Q_{+\infty} = \{r\in \R(X)|r(1/X) \in Q\}$ in $Q_{-\infty} = \{r \in \R(X)| r(-1/X) \in Q\}$. 
\item[(c)] Dokažimo, da je $Q$ edina ureditev obsega $\R(X)$, za katero velja $X\in Q$ in $\epsilon - X \in Q$ za vsak $\epsilon \in \R^{>0}$.
\item[(d)] Dokažimo, da so $Q_{a^+}, Q_{a^-}$, kjer $a\in \R$, ter $Q_{+\infty}, Q_{-\infty}$ vse ureditve obsega $\R(X)$. Dokažimo tudi, da so paroma različne.
\item[(e)] Dokažimo, da je $Q = \{r\in \R(X)| \exists \epsilon \in \R^{>0}: r|_{(0,\epsilon)} \ge 0$. Izpeljimo podobne karakterizacije še za ureditve iz točke (b).
\item[(f)] Določimo oziroma podrobno izpeljimo naravne valuacije ureditev iz točke (b).
\end{enumerate}
Dodatno vprašanje: Naj bo $K$ realno zaprta razširitev urejenega obsega $(\R(X),Q)$. Dokažimo, da $\R(X)$ ni gost v $K$ v intervalski topologiji (namig: ali obstaja tak $a \in \R(X)$, da velja $\sqrt{X} < a < 2\sqrt{X}$?).
\emph{Rešitev}
\begin{enumerate}
\item[(a)] Preverimo najprej, da je $Q+Q \subseteq Q$. Naj bosta $X^m g(x), X^n f(x)\in Q$. Predpostavimo lahko, da je $m\ge n$. Potem je $X^m g(X) + X^n f(X) = X^n (X^{m-n} g(X) + f(X)) \in Q$. Res, velja namreč $n\in \Z$, ker je $m\ge n$, pa je $X^{m-n}g(X) + f(X)$ dobro definiran in strogo pozitiven (ker $f(0)>0$) v $0$. Preverimo sedaj, da je $Q\cdot Q\subseteq Q$. Naj bosta $X^m g(X), X^n f(X) \in Q$. Predpostavimo lahko, da sta neničelna, sicer trditev očitno velja. Zapišemo lahko $X^m g(X) X^n f(X) = X^{m+n} g(X)f(X)$ in ker sta $f(X)$ in $g(X)$ dobro definirana in strogo pozitivna v 0, je tak tudi $g(X)f(X)$. Preverimo še, da je $Q\cap -Q = \{0\}$. Denimo, da je $0\neq X^m g(X) \in Q\cap - Q$. Potem je $m\in\Z$ in $g(X)$ mora biti dobro definiran v $0$, hkrati pa morata biti tako $g(0)$ kot $-g(0)$ strogo pozitivna. Protislovje. Velja, da je $Q$ pozitiven stožec na $\R(X)$. Naj bo sedaj $f(X)\in \R(X)$ poljuben element. Zapišemo ga lahko kot $f(X) = \frac{f_1(X)}{f_2(X)} = \frac{X^m \tilde{f}_1(X)}{X^n \tilde{f}_2(X)}$, kjer sta $\tilde{f}_i(0)\neq 0, i=1,2$. To lahko naprej zapišemo kot $f(X) = X^{m-n} g(X)$, kjer je $g$ dobro definiran neničelen v 0.  Če je $g(0) > 0$, je torej $f\in Q$, če pa je $g(0)< 0$, pa je $f\in -Q$. Sledi, da je $Q\cup -Q = \R(X)$.

\item[(b)] Naj bo $a\in\R$. Pokažimo, da je množica $Q_{a^+}$ ureditev. Naj bosta $r,q\in Q_{a^+}$. To pomeni, da sta $r(X-a),q(X-a)\in Q$. Ker pa je $Q$ ureditev, je tudi $r(X-a) + q(X-a) = (r+q)(X-a)\in Q$ in posledično je $r+q \in Q_{a^+}$. Sledi $Q_{a^+} + Q_{a^+} \subseteq Q_{a^+}$. Naj bo zdaj $r\in Q_{a^+} \cap -Q_{a^+}$. To pomeni, da je $r(X-a) \in Q$ in $-r(X-a) \in Q$. Ker je $Q\cap -Q = 0$, je $r(X-a) = 0$ in posledično še $r=0$. Sledi $Q_{a^+} \cap -Q_{a^+} = \{0\}$. Naj bosta $r,q$ kot zgoraj. Zapišemo lahko $r(X-a) = X^m g(X)$ in $q(X-a) = X^n f(X)$, kjer sta $g$ in $f$ dobro definirana in strogo pozitivna v 0. Potem velja $r(X-a)g(X-a) = X^{m+n} (g(X) f(X))$, kjer je tudi $g(X)f(X)$ dobro definiran in strogo pozitiven v 0. Sledi, da je $r(X-a)q(X-a)\in Q$ in posledično je $Q_{a^+} Q_{a^+} \subseteq Q_{a^+}$. Sledi, da je $Q_{a^+}$ ureditev. Preverimo še linearnost. Naj bo $r\in \R(X)$ poljuben. Poglejmo si $R(X-a)$. Ker je $Q$ linearna, je bodisi $r(X-a) \in Q$ bodisi $r(X-a) \in -Q$. V prvem primeru sledi, da je $r\in Q_{a^+}$, v drugem primeru pa je $r\in -Q_{a^+}$. Sledi $Q_{a^+} \cup -Q_{a^+} = \R(X)$. Na enak način dokažemo željeno še za ostale tri množice.

\item[(c)] Najprej opazimo, da je $X=X\cdot 1$ in da je polinom $\epsilon - X$ dobro definiran in strogo pozitiven v 0 za pozitivne $\epsilon$. To pomeni, da je res $X, \epsilon - X \in Q$.

Opazimo, da iz predpostavk $X\ge 0$ in $\epsilon - X \ge 0 (\Leftrightarrow \epsilon \ge X)$ za vsak $\epsilon > 0$ sledi, da je $\dots > X^{-3} > X^{-2} > X^{-1} > 1 > X > X^2 > X^3> \dots $. Naj bo sedaj $f(X)  \in \R(X)$ oblike $f(X) = \frac{\alpha X^2 + \beta X + \gamma}{\delta X^2 + \epsilon X +\mu}$ in naj bo dobro definiran v 0. Pokažimo, da je njegov predznak odvisen le od predznaka $\frac{\gamma}{\mu}$. Ker velja $X^2 < X < \frac{1}{2a}$ za vsak $a>0$, takoj dobimo neenakost $\frac{1}{2} > aX^2$ ($a>0$). Ta neenakost velja tudi za $a\le 0$, saj je tedaj tudi $aX^2$ negativen. Za vsak $a\in \R$ torej velja $\frac{1}{2} > aX^2$. Podobno lahko ocenimo še, da za vsak $b\in \R$ velja $\frac{1}{2} > bX$. Sedaj lahko ocenimo, da je $aX^2 + bX < \frac{1}{2} + bX < \frac{1}{2} + \frac{1}{2} = 1$, kjer sta $a,b\in \R$ poljubna. Zgornjo neenakost lahko prepišemo v $1 - (aX^2 + bX) > 0$ in ker sta $a,b$ poljubna, je to ekvivalentno temu, da je $1 + aX^2 + bX > 0$. 
Sedaj lahko zapišemo $f(X) = \frac{\alpha X^2 + \beta X + \gamma}{\delta X^2 + \epsilon X +\mu} = \frac{\gamma}{\mu} \cdot  \frac{\alpha/\gamma X^2 + \beta/\gamma X + 1}{\delta/\mu X^2 + \epsilon/\mu X +1}$. Po zgornjem premisleku je  $\alpha/\gamma X^2 + \beta/\gamma X + 1>0$ in $ \delta/\mu X^2 + \epsilon/\mu X +1>0 $. To pa že pomeni, da je predznak $f(X)$ odvisen le od predznaka $\frac{\gamma}{\mu}$. Podobno kot smo zgoraj pokazali, da je vsak polinom druge stopnje, ki ima prosti člen enak 1, pozitiven, lahko pokažemo še, da je poljuben polinom s prostim členom enakim 1 pozitiven (vsak monom je potrebno oceniti, da je manjši kot $\frac{1}{\deg f}$, kar naredimo podobno kot zgoraj). To nam pove, da je predznak kvocienta dveh polinomov, ki imata neničelna prosta člena, odvisen le od predznaka kvocienta teh dveh prostih členov. 

Zgoraj smo videli, da je $X^m, m\in \Z$ pozitiven. Prav tako smo videli, da je $f(X) = \frac{g(X)}{h(X)}, h(0) \neq 0$ pozitiven natanko tedaj, ko je $\frac{g(0)}{h(0)} \ge 0$. Sledi torej, da je $X^m f(X) \ge 0$ natanko tedaj, ko je $f$ dobro definiran in pozitiven v 0. To pa je ravno ureditev $Q$.

\item[(d)] Ločimo nekaj primerov, glede na to, kaj velja za $X$:
\begin{enumerate}
\item $\exists a \in \R$, da za vsak $\epsilon \in \R^{>0}$ velja $a < X < a+ \epsilon$,
\item $\exists a \in \R$, da za vsak $\epsilon \in \R^{>0}$ velja $a-\epsilon < X < a$,
\item $x>\R$,
\item $x<\R$.
\end{enumerate}
V prvem primeru očitno velja $X-a > 0$ in $X-a < \epsilon$. Če označimo z $Y=X-a$, dobimo $Y>0$ in $Y<\epsilon$. Po točki (c) je $Q$ edina ureditev za katero to velja. Sklenemo torej, da je $Y = X-a\in Q$. Dobimo torej ravno ureditev $Q_{a^+}$. Podobno v drugem primeru dobimo, da velja $a-X>0$ in $X-a < \epsilon$. Iz enoličnosti ureditve v kateri velja $X>0, X<\epsilon$ sledi, da smo dobili ureditev $Q_{a^-}$. Pri (c) velja $X>\R$, kar pomeni, da je $X>\R^+$. To je ekvivalentno temu, da je $0 < \frac{1}{X} < \R^+$. Če spet označimo z $Y=\frac{1}{X}$, potem velja $0 < Y < \R^+$, torej je $Y\in Q$. Tako dobimo ravno ureditev $Q_{+\infty}$. V primeru (d) pa velja $X<\R$. To pomeni, da velja še $X< \R^-$, kar je ekvivalentno temu, da je $-X > \R^+$ oziroma $0 < -\frac{1}{X} < \R^+$. Tako dobimo ureditev $Q_{-\infty}$. Ker smo zgoraj obravnavali vse primere, glede na to, kakšen je $X$ v primerjavi z drugimi elementi, smo s tem pokrili vse ureditve. Pri tem smo vsakič dobili ureditev, ki je bila take oblike kot ena izmed štirih zgoraj definiranih. Sledi, da so to vse ureditve.

Pokažimo sedaj, da so ureditve teh štirih oblik paroma različne med seboj.
\begin{itemize}
\item $Q_{+\infty} \neq Q_{-\infty}$, ker v prvi ureditvi $X>0$, v drugi pa $X<0$.

\item $Q_{+\infty}\neq Q_{a^+}$:
\begin{itemize}
\item če $a>0$, je $X\in Q_{+\infty}$; po drugi strani pa je $X\in Q_{a^+}\Leftrightarrow X-a \in Q\Leftrightarrow -a >0$, kar pa ne velja, torej $X\notin Q_{a^+}$,
\item če $a=0$, potem je $Q_{a^+} = Q$ in po (c) vsebuje $\epsilon - X$ za neki $\epsilon > 0$. Po drugi strani pa je $\epsilon - X 	\in Q_{+\infty}\Leftrightarrow \epsilon - \frac{1}{X} = \frac{\epsilon X - 1}{X} \in Q \Leftrightarrow - 1>0$, kar pa spet ne drži, torej $\epsilon - X \notin Q_{+\infty}$
\item če $a<0$, potem je $X+\frac{3}{2} a \in Q_{+\infty}$ in $X+\frac{3}{2} a \notin X_{a^+}$ (To pokažemo podobno kot zgoraj, podobno bo veljalo v naslednjih točkah, zato bom te premisleke spuščal),
\end{itemize}

\item $Q_{+\infty} \neq Q_{a^-}$
\begin{itemize}
\item če $a\le 0$, je $X\in Q_{+\infty}$ in $X\notin Q_{a^-}$,
\item če $a>0$, je $2a-X \in Q_{a^-} $ in $2a-X\notin Q_{+\infty}$, 
\end{itemize}

\item $Q_{-\infty} \neq Q_{a^+}$
\begin{itemize}
\item če $a<0$, potem je $-X\in Q_{-\infty}$ in $-X\notin Q_{a^+}$,
\item če je $a=0$, potem je $X\in Q_{a^+}$ in $X\notin Q_{-\infty}$,
\item če je $a>0$, potem je $X+2a \in Q_{a^+}$ in $X+2a \notin Q_{-\infty}$,
\end{itemize}

\item $Q_{-\infty}  \neq Q_{a^-}$
\begin{itemize}
\item če $a>0$, je $X\in Q_{a^-}$ in $X\notin Q_{-\infty}$,
\item če $a=0$, je $1+X\in Q_{a^-}$ in $1+X\notin Q_{-\infty}$,
\item če $a<0$, je $-2a + X\in Q_{a^-}$ in $-2a + X\notin Q_{-\infty}$,
\end{itemize}

\item $Q_{a^+} \neq Q_{a^-}$
\begin{itemize}
\item če $a=0$, potem je $X\in Q_{a^+}$ in $X\notin Q_{a^-}$,
\item če $a>0$, potem je $X\in Q_{a^-}$ in $X\notin Q_{a^+}$,
\item če $a<0$, potem je $X\in Q_{a^+}$ in $X\notin Q_{a^-}$,
\end{itemize}

\item $Q_{a^+} \neq Q_{b^+}$ za $a\neq b$. Velja bodisi $a>b$ bodisi $b>a$. Predpostavimo lahko, da je $a>b$, sicer ju preimenujemo. Potem je $b+X\in Q_{b^+}$, ker je $b+(X-b) = X = X\cdot 1 \in Q$ in $b+X\notin Q_{a^+}$, ker $b-a\not > 0$.


\item $Q_{a^-} \neq Q_{b^-}$ za $a \neq b$. Spet lahko predpostavimo, da je $a>b$. Potem je $b-X\in Q_{b^-}$ in $b-X\notin Q_{a^-}$.

\end{itemize}
KAJ PA PRIMER $Q_{a^+} \neq Q_{b^-}$

MOGOČE TOLE NI RAVNO NAJBOLJE


\item[(e)]

\item[(f)]

\end{enumerate}

\begin{flushleft}
9. naloga
\end{flushleft}
Naj bo $K$ realno zaprt obseg in $v$ valuacija na $K$, ki je kompatibilna z edino ureditvijo na $K$. 
\begin{enumerate}
\item[(a)] Dokažimo, da je tudi residualni obseg $k_v = A_v / I_v$ realno zaprt.
\item[(b)] Dokažimo, da je valuacijska grupa $\Gamma_v$ \emph{deljiva}, se pravi, da za vsak $u \in \Gamma_v$ in vsak $n\in \N$ obstaja tak $u' \in \Gamma_v$, da je $u = nu'$.
\item[(c)] Dokažimo, da za vsak polinom $f=x^n + a_{n-1} x^{n-1} + \dots a_1 x + a_0 \in A_v[x]$ in za vsako ničlo $\overline{a} \in k_v$ polinoma  $\overline{f} =x^n + \overline{a}_{n-1} x^{n-1} + \dots +\overline{a}_1 x + \overline{a}_0 \in k_v[x]$ obstaja taka ničla $b\in A_v$ polinoma $f$, da je $\overline{b} = \overline{a}$.
\end{enumerate}
\emph{Rešitev}
\begin{enumerate}
\item[(a)]
\item[(b)]
\item[(c)]
\end{enumerate}

\begin{flushleft}
10. naloga
\end{flushleft}
Naj bo $K$ urejen obseg. Dokažimo:
\begin{enumerate}
\item[(a)] $K$ je arhimedski natanko tedaj, ko je grupa $K^{>0}$ arhimedska.
\item[(b)] valuacija $v$ na $K$ je kompatibilna z ureditvijo na $K$ natanko tedaj, ko je množica $\{a\in K| v(a) = 0 \text{ in } a > 0\}$ konveksna podgrupa v $K^{>0}$.
\end{enumerate}
\emph{Rešitev}
\begin{enumerate}
\item[(a)] Naj bo najprej grupa $K^{>0}$ arhimedska. Pokažimo, da je $K$ arhimedski. To je ekvivalentno temu, da $K$ nima infinitezimalnih elementov. Denimo nasprotno, da ima infinitezimalni element $a \neq 0$. To pomeni, da za vsako naravno število $n$ velja $n|a| \le 1$. Izberimo zdaj naravno število $k > 1$. Potem zgornji pogoj pove, da za vsako naravno število $m$ velja $k^m |a| \le 1$. Ker so elementi $|a|, k^m > 0$, ta zveza velja tudi v multiplikativni grupi $K^{>0}$. To pa je ekvivalentno s tem, da v $K^{>0} $ za vsako naravno število $m$ velja $k^m \le |a|^{-1}$. Po krepki arhimedskosti grupe $K^{>0}$, ki je za $\ell$-grupe ekvivalentna arhimedskosti, potem sledi, da je $k\le 1$, kjer je $1$ enota $K^{>0}$. Protislovje.

Naj bo zdaj še $K$ arhimedski. Po H\"{o}lderjevem izreku lahko $K$ urejenostno vložimo v $\R$ (označimo z $\varphi$ to vložitev). Potem je grupa $K^{>0}$ izomorfna neki podgrupi $(\R^{>0}, \cdot, \le)$. Vzemimo poljubna pozitivna elementa $a,b$, za katera velja $a^n \le b$, kjer je $n\in\Z$ poljuben. Potem za vsako celo število $n$ velja $ \varphi(a)^n =  \varphi(a^n) \le \varphi(b)$. To pa že pomeni, da je $\varphi(a) = 1$, torej je tudi $a=1$. To pa ravno pomeni, da je $K^{>0}$ arhimedska grupa.
\item[(b)]
\end{enumerate}

\begin{flushleft}
11. naloga
\end{flushleft}
Dokaži, da vsak algebraično zaprt obseg $K$ s karakteristiko 0 vsebuje tak realno zaprt obseg $R$, da velja $K = R[i]$.
\newline
\emph{Rešitev}
\newline


\end{document}