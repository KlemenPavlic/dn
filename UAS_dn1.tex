% Začetek preambule
\documentclass[a4paper, 12pt]{article}
\usepackage[slovene]{babel}
\usepackage[utf8]{inputenc}
\usepackage[T1]{fontenc}
\usepackage{lmodern}
\usepackage{amsfonts,amsmath,amssymb}
\usepackage{graphicx}


% Moji ukazi, okolja,...
% oznake za števila
\newcommand{\N}{\mathbb{N}}
\newcommand{\Z}{\mathbb{Z}}
\newcommand{\Q}{\mathbb{Q}}
\newcommand{\R}{\mathbb{R}}
\newcommand{\C}{\mathbb{C}}
\newcommand{\F}{\mathbb{F}}
% opombe
\newenvironment{opomba}{\begin{flushleft} \textbf{Opomba}:}{\hfill \end{flushleft}}
% definicije
\newcounter{definitionCounter}
\addtocounter{definitionCounter}{1}
\newenvironment{definicija}{\begin{flushleft} \textit{\textbf{Definicija \arabic{definitionCounter}}}:}{\hfill \end{flushleft}\stepcounter{definitionCounter}}
% pojmi
\newcommand{\pojem}[1]{\textsc{#1}}
% dokazi
\newenvironment {dokaz}{\begin{flushleft} \textit{\textbf{Dokaz}}:}{\hfill $\square$\end{flushleft}}
% izreki
\newcounter{theoremCounter}
\addtocounter{theoremCounter}{1}
\newcounter{theoremCorollaryCounter}
\addtocounter{theoremCorollaryCounter}{0}
\newenvironment {izrek}{\begin{flushleft} \textsf{\textbf{IZREK \arabic{theoremCounter}}}:}{\hfill \end{flushleft}\stepcounter{theoremCounter}\stepcounter{theoremCorollaryCounter}\setcounter{corollaryCounter}{1}}
% leme
\newcounter{lemmaCounter}
\addtocounter{lemmaCounter}{1}
\newenvironment{lema}{\begin{flushleft} \textbf{Lema \arabic{lemmaCounter}}:}{\hfill \end{flushleft}\stepcounter{lemmaCounter}}
% posledice
\newcounter{corollaryCounter}
\addtocounter{corollaryCounter}{1}
\newenvironment  {posledica}{\begin{flushleft} \textsf{\textbf{Posledica \arabic{theoremCorollaryCounter}.\arabic{corollaryCounter}}}:}{\hfill \end{flushleft}\stepcounter{corollaryCounter}}
% dodatni ukazi
\usepackage{hyperref} % mora biti zadnji

% začetek dokumenta
\begin{document}
\thispagestyle{empty}
\noindent{\large
UNIVERZA V LJUBLJANI\\[1mm]
FAKULTETA ZA MATEMATIKO IN FIZIKO\\[5mm]
MATEMATIKA -- 2.~stopnja}
\vfill

\begin{center}{\large 
Klemen Pavlič\\[2mm]
{\bf Urejenostne algebrske strukture}\\[10mm]
Domača naloga}
\end{center}
\vfill

\noindent{\large
Velika Lašna, 2013}
\pagebreak
\newpage 

\begin{flushleft}
1. naloga
\end{flushleft}
Naj bo $G=\Z\times \Z$ z operacijo $(a,b)\circ (c,d) = (a+(-1)^b c, b+d)$.
\begin{enumerate}
\item[(a)] Dokažimo, da je $G$ grupa in da nima nobene linearne  ureditve.
\item[(b)] Dokažimo, da $G$ nima nobene mrežne ureditve.
\item[(c)] Konstruiraj na $G$ kako usmerjeno delno ureditev.
\end{enumerate}
\emph{Rešitev}
\begin{enumerate}
\item[(a)] Preverimo najprej, da je grupa. Jasno je, da je množica $\Z \times \Z$ zaprta za $\circ$. Preverimo asociativnost.
$$
\big[(a,b)\circ (c,d)\big] \circ (e,f) = (a+(-1)^b c, b+d) \circ (e,f) =
$$
$$
(a+(-1)^b c+ (-1)^{b+d}e, b+d+f),
$$
$$
(a,b)\circ \big[ (c,d) \circ (e,f) \big] = (a,b) \circ (c+(-1)^d e, d+f) =
$$
$$
(a+(-1)^b (c+(-1)^de), b+(d+f)) = (a+ (-1)^b c + (-1)^{b+d} e, b+d+f).
$$
Element $(0,0)$ je enota. Res, velja namreč
$$
(a,b)\circ (0,0) = (a+(-1)^b 0, b+0) = (a,b),
$$
$$
(0,0) \circ (a,b) = (0+(-1)^0 a, 0+b) = (a,b). 
$$
Inverz elementa $(a,b)$ je element $(a(-1)^{1-b},-b)$. Res, velja namreč
$$
(a,b) \circ (a(-1)^{1-b},-b) = (a+(-1)^b a (-1)^{1-b}, b-b) = (a-a,0 )= (0,0),
$$
$$
(a(-1)^{1-b}, -b) \circ (a,b) = (a(-1)^{1-b}+(-1)^{-b}a,-b+b)=
$$
$$
(a(-1)^{-b}(-1+1),0 ) = (0,0).
$$
Sledi, da je $(G,\circ)$ res grupa.

Pokažimo še, da $G$ nima nobene linearne ureditve. Denimo nasprotno, da na $G$ obstaja linearna ureditev. Potem za pozitiven stožec $P$ te ureditve velja $P\cup P^{-1} = G$. Preprost račun nam pokaže, da velja
$$
(e,f) \circ (a,b) \circ (e,f)^{-1} = (e+(-1)^f a + (-1)^{1+b} e,b).
$$
Vzemimo sedaj poljuben element oblike $(a,0), a\neq 0$. Njegov inverz je $(-a,0)$. Po drugi strani pa iz zgornje formule sledi, da je $(0,1)\circ(a,0)\circ (0,1)^{-1} = (-a,0)$. Vemo, da je bodisi $(a,0) \in P$ bodisi $(-a,0)\in P$. Predpostavimo lahko, da je $(a,0) \in P$, sicer ga zamenjamo z inverzom.  To pa pomeni, da je element $(-a,0) \in P\cap P^{-1} = \{e\} = \{(0,0)\}$. Sledi, da je $a=0$. Protislovje.

\item[(b)] Denimo nasprotno, da na $G$ obstaja mrežna ureditev. Naj bo spet $P$ njen pozitiven stožec.

\item[(c)] Definirajmo pozitiven stožec $P_{\ge} = \{(a,b); b > 0, a \text{ poljuben}\} \cup \{(0,0)\}$. Preverimo najprej, da je to res pozitiven stožec. Najprej preverimo, da velja $P_{\ge} \circ P_{\ge} \subseteq P_{ \ge}$. Naj bosta $(a,b), (c,d) \in P_{\ge}$. Če sta $b,d > 0$, potem je druga komponenta $(a,b)\circ (c,d)$ enaka $b+d > 0$ in je $(a,b)\circ (c,d) \in P_{\ge}$. Če je $d=0$, mora biti tudi $c=0$. Ker je $(0,0)$ enota, je $(a,b)\circ (0,0) = (a,b)$ in ima drugo komponento pozitivno, torej je v pozitivnem stožcu $P_{\ge}$. Enako velja za primer, ko $b=0$ in $d>0$. Če pa sta $b=d=0$, mora biti še $a=c=0$ in je $(a,b)\circ(c,d) = (0,0) \in P_{\ge}$. Sledi $P_{\ge} \circ P_{\ge} \subseteq P_{\ge}$.

Pokažimo, da je $P_{\ge} \cap P_{\ge} = \{(0,0)\}$. Naj bo element $(a,b)$ v preseku. Potem mora veljati $(a,b) \ge 0$ in $(a,b)^{-1}=(a(-1)^{1-b},-b)\ge 0$. To je res natanko tedaj, ko je bodisi $b>0$ in $-b>0$ bodisi $a=b=0$ in $a(-1)^{1-b}=-b=0$. V prvem primeru pridemo do protislovja, torej mora veljati drugi primer, to pa pomeni, da je $a=b=0$ in da je v preseku res le enota.

Naj bo zdaj še $(c,d)\in G$ poljuben element in $(a,b)\in P_{\ge}$ poljuben element. Preverimo, da je $(c,d)\circ (a,b) \circ (c,d)^{-1} \in P_{\ge}$. Velja
$$
(c,d)\circ (a,b) \circ (c,d)^{-1} = (c,d) \circ (a,b) \circ (c(-1)^{1-d},-d) =
$$
$$
(c+(-1)^d a ,d+b) \circ (c(-1)^{1-d},-d)=
$$
$$
(c+(-1)^d a +(-1)^{b+d}c(-1)^{1-d},d+b-d)=
$$
$$
(c+(-1)^d a + (-1)^{1+b}c,b).
$$
Če je $b>0$ , potem je ta element iz $P_{\ge}$. Če pa je $b=0$, je tudi $a=0$ in zgornje naprej poenostavimo v $(c+(-1)^d a + (-1)^{1+b}c,b) = (c+(-1)c,0)=(0,0) \in P_{\ge}$. Sklenemo, da je $P_{\ge}$ res pozitiven stožec.

Pokažimo sedaj, da je $G$ v tej ureditvi usmerjena. Naj bosta torej $(a,b), (c,d)\in G$ poljubna elementa. Poiskati moramo njuno spodnjo in zgornjo mejo. Trdim, da je element $(0,\max\{b,d\} + 1)$ zgornja meja. Preverimo, da je $(a,b)\le (0,\max\{b,d\} + 1)$. To bo res natanko tedaj, ko bo $(0,\max\{b,d\} + 1)\circ (a,b)^{-1} = (0,\max\{b,d\} +1)\circ (a(-1)^{1-b},-b)=((-1)^{\max\{b,d\}+1} a (-1)^{1-b},\max\{b,d\} - b +1 ) \in P_{\ge}$. To pa velja, ker je druga komponenta $\max\{b,d\} - b +1 \ge 1 > 0$. Podobno vidimo, da je zgornji element tudi zgornja meja za $(c,d)$.

Trdim, da je element $(0,\min\{b,d\} - 1)$ spodnja meja za $(a,b)$ in $(c,d)$. Preverimo, da je $(0,\min\{b,d\} - 1) \le (a,b)$. Podoben račun kot zgoraj pokaže, da je to res natanko tedaj, ko je $(a,b)\circ (0,\min\{b,d\} -1)^{-1} = (a,b+1-\min\{b,d\}) \in P_{\ge}$. To pa spet velja, ker je druga komponenta $b+1-\min\{b,d\} \ge 1 > 0$. Enako se pokaže še $(0,\min\{b,d\} - 1) \le  (c,d)$. Sledi, da je $G$ usmerjena.
\end{enumerate}

\begin{flushleft}
2. naloga
\end{flushleft}
Naj bo $N$ podgrupa edinka grupe $G$ in naj bosta $N$ in $G/N$ delno urejeni. Dokažimo, da je množica 
$$
P:= N^+ \cup \{x\in G | N \neq xN \in (G/N)^+ \}
$$
pozitiven stožec neke delne ureditve na $G$ natanko tedaj, ko velja $gN^+g^{-1} \subseteq N^+$ za vse $g\in G$. V nadaljevanju predpostavimo, da je $P$ pozitiven stožec neke delne ureditve na $G$. Dokažimo:
\begin{enumerate}
\item[(a)] $(N,N^+)$ je urejenostno konveksna podgrupa v $(G,P)$.
\item[(b)] Če $g\in P\setminus N$, potem je $g > N$.
\item[(c)] Grupa $(G,P)$ je linearno urejena natanko tedaj, ko sta grupi $(N,N^+)$ in $(G/N, (G/N)^+)$ linearno urejeni.
\item[(d)] Če je $N\neq \{e\}$, potem je $(G,P)$ mrežno urejena grupa natanko tedaj, ko je $(N,N^+)$ mrežno urejena grupa in je $(G/N, (G/N)^+)$ linearno urejena grupa.
\item[(e)] Če je $(G,P)$ mrežno urejena grupa, potem je vložitev $N$ v $G$ pravilna (tj. ohranja tudi neskončne supremume in infimume).
\end{enumerate}
\emph{Rešitev}
Naj bo najprej $P$ pozitiven stožec neke ureditve na $G$ in $g\in G$ poljuben element. Ker je $N$ edinka je $gN^+ g^{-1} \subseteq N$ in ker je $P$ pozitiven stožec, je $gN^+ g^{-1} \subseteq N^+$.

Dokažimo še obrat. Naj za vsak $g\in G$ velja $g N^+ g^{-1} \subseteq N^+$. Dokažimo, da je $P$ pozitiven stožec. Pokažimo najprej, da je $P\cdot P \subseteq P$. Če $s,b\in N^+$, je $ab \in N^+$, ker je to pozitiven stožec. Če $x,y\in G$ taka, da je $N\neq xN \in (G/N)^+$ in $N \neq yN \in (G/N)^+$, je spet $(xy)N = (xN) (yN)\in (G/N)^+$, ker je to pozitiven stožec, ker pa je produkt strogo pozitivnih elementov strogo pozitiven, je še $(xy)N \neq N$. Če pa je $a\in N^+\subseteq N$ in $x\in G$ tak, da je $N\neq xN \in (G/N)^+$, potem je $(xa)N = (xN) (aN) = (xN)N = xN\in (G/N)^+ \setminus N$. Preverimo sedaj, da je $P\cap P^{-1} = \{e\}$. Velja $P^{-1} = N^- \cup \{ x\in G| N \neq xN \in (G/N)^-\}$. Jasno je $N^+ \cap N^- = \{e\}$ in ker velja tudi $(G/N)^+ \cap (G/N)^- = N$, sledi še $\{x\in G| N \neq xN \in (G/N)^+\} \cap \{x\in G| N \neq xN \in (G/N)^-\} = \emptyset$. Ker sta $N^+, N^- \subseteq  N$, velja še $N^+ \cap \{x\in G|  N \neq xN \in (G/N)^-\} = \emptyset$ in $N^- \cap \{x\in G|  N \neq xN \in (G/N)^+\} = \emptyset$. Sledi $P\cap P^{-1} = \{e\}$. Preveriti moramo še, da je za vsak $g\in G$ $gPg^{-1} \subseteq P$. Po predpostavki je $gN^+ g^{-1} \subseteq N^+$. Če pa je $x\in G$ tak, da je $N\neq xN \in (G/N)^+$, potem je $gxg^{-1} N = (gN) (xN) (gN)^{-1} \in (G/N)^+$, ker je to pozitiven stožec. Jasno je tudi, da je $(gN)(xN)(gN)^{-1} \neq N$, saj bi v nasprotnem priemru sledilo $xN = N$, kar pa ne velja. Sledi torej, da je tudi $gxg^{-1}$ tak, da je $N\neq (gxg^{-1})N \in (G/N)^+$. Zaključimo, da je $P$ res pozitiven stožec.
\begin{enumerate}
\item[(a)]Zadošča pokazati, da je za vsak $x\in N$ interval $[e,x]_G\subseteq N$. Naj bo torej $x\in N$, $y\in G$ in naj velja $e\le y\le x$. Ker je $y\ge e$, je $y\in P$. Če je $y\in N^+ \subseteq N$, smo končali. Predpostavimo torej lahko, da je $y\in \{x\in G| N \neq xN \in (G/N)^+\}$. To pomeni, da velja $yN\neq N$, od koder sledi tudi $y^{-1}N \neq N$, zato je $y^{-1} \in \{ x\in G| N \neq xN \in (G/N)^-\}$. To pa pomeni, da za vsak $z \in N$ velja $y^{-1} z \notin N$. Ker je $y\le x$, je $xy^{-1} \in P$ in posledično je $y^{-1} xy^{-1} y = y^{-1} x \in P$. Po zgornjem je $y^{-1} x \in \{z\in G| N\neq zN \in (G/N)^+\}$. To pa pomeni, da je $y^{-1}xN \in (G/N)^+ \setminus N$. Po drugi strani pa je $y^{-1} x N = (y^{-1}N ) (xN) = (y^{-1} N)N = y^{-1}N \in (G/N)^- \setminus N$. Sledi, da je $y^{-1} xN \in \big((G/N)^+ \setminus N\big) \cap \big( (G/N)^- \setminus N\big)=\emptyset$. Protislovje. Zaključimo, da je $y\in N$ in zato $N$ urejenostno konveksna.

\item[(b)] Naj bo $g\in P\setminus N$. Pokažimo, da je $g> N$. To je ekvivalentno temu, da za vsak $x\in N$ velja $g> x$. Denimo nasprotno, da obstaja $x\in N$, da je $g\le x$. Ker je $g\in P$, je $e \le g$. Velja torej $e \le g \le x$  oziroma $g\in [e,x]_G$  in ker je  $N$ urejenostno konveksna, to implicira, da je $g\in N$. Protislovje.

\item[(c)] Denimo najprej, da sta $N$ in $G/N$ linearno urejeni. Po predpostavki je $P$ pozitiven stožec. Pokažimo, da velja $P\cup P^{-1} = G$. Če je $g\in N$, potem zaradi linearne urejenosti slednje velja $g\in N^+ \cup N^-$. Če pa $g\notin N$, potem je $gN\in (G/N)\setminus N= \big((G/N)^+ \setminus N\big) \cup \big((G/N)^- \setminus N\big)$, ker je $G/N$ linearno urejena. Sledi, da za vsak $g\in G$ velja $g\in P\cup P^{-1}$ in zato je  $P\cup P^{-1} = G$, torej je $G$ linearno urejena.

Naj bo zdaj še $G$ linearno urejena. Pokažimo, da sta $N$ in $G/N$ linearno urejeni. Velja, da je $N\cap P$ pozitiven stožec na $N$. Res, velja namreč $(N\cap P)(N\cap P)\subseteq N\cap P$, ker $N$ podgrupa in $P$ pozitiven stožec. Prav tako je $(N\cap P) \cap (N\cap P)^{-1}  = (N\cap P ) \cap (N\cap P^{-1}) = N \cap P \cap P^{-1} = N\cap \{e\} = \{e\}$. Ker $N$ edinka in $P$ pozitiven stožec je tudi $g(N\cap P) g^{-1} \subseteq N\cap P$ za vsak $g\in G$. To je torej res pozitiven stožec. Ker pa je $P\cup P^{-1} = G$ in je $N\subseteq G$, je $(N\cap P) \cup (N\cap P^{-1})$ pokritje za $N$ in je zato $N$ linearno urejena.

Naj bo $\pi : G\rightarrow G/N$ urejenostni homomorfizem. Ker je $N$ po točki (a) urejenostno konveksna, je po izreku s predavanj množica $\pi(P)$ pozitiven stožec na $G/N$. Ker je $\pi$ homomorfizem, je $(\pi(P))^{-1} = \pi (P^{-1})$. Naj bo zdaj $N\neq gN \in G/N$ poljuben. Potem obstaja $h \in G\setminus N$, da je $\pi(h) = gN$. Ker je $G= P\cup P^{-1}$, je bodisi $h\in P\setminus N$ bodisi $h\in P^{-1} \setminus N$. V prvem primeru je po točki (b) element $h> N$ in ker je $\pi$ urejenotsni homomorfizem, je $\pi(h) = gN > N$, torej je $gN \in \pi(P)$. V drugem primeru pa je element $h^{-1} \in P\setminus N$ in je zato $h^{-1}> N$. Spet sledi, da je $(gN)^{-1} = (\pi(h))^{-1} = \pi(h^{-1}) > N$ in zato $gN \in (\pi(P))^{-1}$. Preverimo le še, da se ne more zgoditi, da bi obstajala $h_1,h_2\in G\setminus N$, ki bi zadoščala $\pi(h_i) = gN$ in $h_1 \in P, h_2 \in P^{-1}$. Recimo, da velja nasprotno. Ker je $h_1 \in P\setminus N$, je po (b) $h_1 > N$. Ker je $h_2 \in P^{-1}\setminus N$, je $h_2^{-1} > N$. Potem velja $h_1 h_2^{-1} > h_2^{-1} N^+ \ge N^+$ in zato je $h_1 h_2^{-1}> N^+$. Ker je $h_1h_2^{-1}\in P$, je še $h_1 h_2^{-1} > N^-$. Sledi, da je $h_1 h_2^{-1}>N$. Ker pa velja $\pi(h_1) = gN = \pi(h_2)$, mora biti $h_1 h_2^{-1}\in N$. Protislovje s tem, da je $h_1h_2^{-1}>N$. S tem smo pokazali, da je $\pi(P) \cup (\pi(P))^{-1} = G/N$ in je zato tudi $G/N$ linearno urejena.

\item[(d)]
\item[(e)]
\end{enumerate}

\begin{flushleft}
3. naloga
\end{flushleft}
Dokažimo, da ima vektorski prostor $\mathcal{C}^{(1)} [a,b]$ vseh zvezno odvedljivih funkcij na intervalu $[a,b]$ (z urejenostjo $f\ge g \Leftrightarrow f(x) \ge g(x)$ za vsak $x\in [a,b]$) Rieszovo interpolacijsko lastnost.
\newline
\emph{Opomba}
\newline
Rešitev te naloge je v članku "The Riesz Interpolation Property for the Space of Continously Differentiable Functions", ki ste nama ga dali na predavanjih. 
Premislimo torej, kaj se poenostavi, če ne zahtevamo zvezne odvedljivosti. Obdelajmo še rešitev iz Lavričeve knjige, kjer dela z dekompozicijsko lastnostjo.
\newline
\emph{Rešitev}

\begin{flushleft}
4. naloga
\end{flushleft}
Naj bo $I$ neskončna množica.
\begin{enumerate}
\item[(a)] Dokažimo, da je $\oplus_{i\in I} \Z$ strnjena podgrupa v $\prod_{i\in I} \Z$.
\item[(b)] Dokažimo, da faktorska grupa $\prod_{i\in I} \Z / \oplus_{i\in I} Z$ ni arhimedska.
\end{enumerate}
\emph{Rešitev}
\begin{enumerate}
\item[(a)] Elementi $\oplus_{i\in I}$ imajo le končno mnogo neničelnih komponent. Jasno je, da je $\oplus_{i \in I} \Z$ podgrupa, saj je vsota dveh elementov, ki imata le končno neničelnih komponent, tudi sama take oblike. Isto velja za inverze. Pokažimo raje, da je strnjena. Naj bo torej $a = (a_i)_{i\in I}\in \oplus_{i \in I} \Z$ in $b = (b_i)_{i\in I}\in \prod_{i\in I} \Z$. Naj zanju velja še $|b| \le |a|$. To pomeni, da za vsak $i\in I$ velja $|b_i| \le |a_i|$. Naj bo sedaj $A= \{ i \in I; a_i \neq 0\}$ in $A^C = \{i\in I; a_i =0\}$. Po predpostavki je $|A| < \infty$. Iz zveze $|b_i| \le |a_i| $ pa sledi še, da je za vsak $i\in A^C$ element $b_i=0$. Sledi, da ima $b$ lahko neničelne komponente le na mestih, na katerih so indeksi iz množice $A$. To pa pomeni, da ima tudi $b$ lahko največ končno mnogo neničelnih komponent in je zato tudi $b\in \oplus_{i\in I} \Z$. Zaključimo, da je $\oplus_{i\in I}\Z$ strnjena podgrupa v $\prod_{i \in I} \Z$.
\item[(b)] Premislimo najprej, kdaj v faktorski grupi $\prod_{i\in I} \Z/ \oplus_{i\in I} \Z$ enačimo dva elementa. Po definiciji je $a + \oplus_{i\in I} \Z = b + \oplus_{i\in I} \Z $ natanko tedaj, ko je $a-b\in \oplus_{i\in I} \Z$. To pa velja natanko tedaj, ko ima $a-b$ le končno mnogo neničelnih komponent oziroma ko se $a$ in $b$ razlikujeta le na končno mnogo komponentah. Od tod takoj sledi, da so v ekvivalenčnem razredu enote natanko vsi tisti elementi, ki imajo le končno neničenih komponent. Prav tako velja še, da je $a\le b$ v faktorski grupi natanko tedaj, ko ima $b-a$ le končno negativnih komponent. Vzemimo sedaj elementa $a=(1,1,1,\dots)$ in $b=(1,2,3,\dots)$. Potem je $b-na = (1-n,2-n,3-n,\dots, -2,-1,0,1,2,3,\dots)$. Sledi, da ima ta  razlika za vsak $n$ le končno mnogo negativnih komponent in je zato  pozitiven element. To je ekvivalentno temu, da je za vsak $n\in \N$ veljavna zveza $na \le b$, vendar pa je $a\neq 0$, saj se razlikueta v vseh komponentah. Sledi, da faktorska grupa $\prod_{i\in I}\Z/ \oplus_{i\in I} \Z$ res ni arhimedska.
\end{enumerate}

\begin{flushleft}
5. naloga
\end{flushleft}
Naj bo $G$ aditivna  grupa vseh celoštevilskih $2\times 2$ matrik z delno urejenostjo
$$
\begin{bmatrix}
a_1 & b_1 \\
c_1 & d_1 \\
\end{bmatrix}
\le 
\begin{bmatrix}
a_2 & b_2 \\
c_2 & d_2 \\
\end{bmatrix}
\Leftrightarrow
\begin{array}{l}
a_1 < a_2 \text{ ali}\\
a_1 = a_2 \text{ in } b_1 < b_2 \text{ ali}\\
a_1 = a_2 \text{ in } b_1 = b_2 \text{ in } c_1 \le c_2 \text{ in } d_1 \le d_2\\
\end{array}.
$$
\begin{enumerate}
\item[(a)] Dokaži, da je $G$ mrežno urejena.
\item[(b)] Izračunaj vrednosti koordinatnih matrik $E_{ij}$.
\item[(c)] Dokaži, da smo v (b) dobili vse prapodgrupe  v $G$.
\item[(d)] Opiši Hahnovo vložitev za grupo $G$.
\end{enumerate}
\emph{Rešitev}
\begin{enumerate}
\item[(a)]Trdim, da je množica
$$
P=\big\{
\begin{bmatrix}
a_1 & b_1 \\
c_1 & d_1 \\
\end{bmatrix}
;
\begin{array}{l}
a_1 >0 \text{ ali}\\
a_1 = 0 \text{ in } b_1 > 0 \text{ ali}\\
a_1 = 0 \text{ in } b_1 = 0 \text{ in } c_1 \ge 0 \text{ in } d_1 \ge 0\\
\end{array}
\big\}
$$
pozitiven stožec. Če ločimo nekaj primerov, z lahkoto pokažemo, da je $P+P \subseteq P$. Pokažimo še, da je $P\cap -P = \{0\}$. Če je 
$\begin{bmatrix}
a_1 & b_1 \\
c_1 & d_1\\
\end{bmatrix}
$
iz $P\cap -P$, potem mora veljati $a_1 > 0$ ali $(a_1 = 0 \text{ in } b_1 > 0)$ ali $(a_1=0 \text{ in } b_1 = 0 \text{ in } c_1 \ge 0 \text{ in } d_1 \ge 0)$ in hkrati še  $-a_1 > 0$ ali $(-a_1 = 0 \text{ in } -b_1 > 0)$ ali $(-a_1=0 \text{ in } -b_1 = 0 \text{ in } -c_1 \ge 0 \text{ in } -d_1 \ge 0)$. Sledi, da mora veljati $a_1 =b_1 =0$ in $0\le c_1 \le 0, 0 \le d_1 \le 0$, torej še $c_1 = d_1 = 0$. Sledi, da je v preseku res samo enota in ker je ta grupa komutativna, je ta množica res pozitiven stožec. 
Pokažimo še, da je mreža v tej urejenosti. Naj bosta 
$
A=
\begin{bmatrix}
a_1 & b_1 \\
c_1 & d_1 \\
\end{bmatrix}
$
in
$
B = 
\begin{bmatrix}
a_2 & b_2 \\
c_2 & d_2 \\
\end{bmatrix}
$
poljubni matriki. Poiščimo njun supremum. Če je $a_1 > a_2 $, potem je $A \lor B = A$ in podobno  je v primeru  $a_1 < a_2$ $A\lor B = B$. Naj bo torej zdaj $a_1 = a_2$. Če je še $ b_1 < b_2 $ je $A\lor B = B$, če pa je $b_1 > b_2$, pa je $A \lor B = A$. Opazujmo zdaj primer, ko je $a_1 = a_2 $ in $b_1 = b_2$. Ločimo še nekaj primerov. Če je 
\begin{itemize}
\item $c_1 \ge c_2 $ in $d_1\ge d_2$ je $A\lor B = A$,
\item $c_1 \le c_2$ in $d_1 \le d_2$ je $A\lor B = B$,
\item $c_1 \ge c_2$ in $d_2 \ge d_1$ je  $A\lor B=
\begin{bmatrix}
a_1 & b_1\\
c_1 & d_2 \\
\end{bmatrix}$,
\item $c_2 \ge c_1$ in $d_1 \ge d_2$ je  $A\lor B=
\begin{bmatrix}
a_1 & b_1\\
c_2 & d_1 \\
\end{bmatrix}$.
\end{itemize}
Vsi ti elementi so res pravi supremumi. Podobno poiščemo še infimume. Sledi, da je $G$ v dani urejenosti res mreža.
\item[(b)] Določimo sedaj vrednosti koordinatnih matrik $E_{ij}$. Vemo, da je vrednost maksimalna strnjena podgrupa, ki še ne vsebuje danega elementa. Začnimo z elementom $E_{11}$. Trdim, da je 
$$
V_{E_{11}} = 	\big\{ 
\begin{bmatrix}
0 & b_1 \\
c_1 & d_1 \\
\end{bmatrix}; b_1,c_1,d_1 \in \Z
  \big\}.
$$
Očitno je $V_{E_{11}}$ podgrupa in $E_{11}\notin V_{E_{11}}$. Preverimo, da je $V_{E_{11}}$ strnjena. Denimo, da za matriki 
$B=\begin{bmatrix}
a_2 & b_2\\
c_2 & d_2\\
\end{bmatrix}
\in G
$
in
$
A=\begin{bmatrix}
0 & b_1\\
c_1 & d_1\\
\end{bmatrix}
\in V_{E_{11}}
$
velja $|B| \le |A|$. To je ekvivalentno temu, da je $|a_2| < 0 $ ali $|a_2| = 0$ in $|b_2| < |b_1|$ ali $|a_2| = 0$ in $|b_2| = |b_1| $ in $|c_2| \le |c_1|$ in $|d_2| \le |d_1|$. Jasno je, da prvi pogoj ne more biti izpolnjen, kar pomeni, da mora biti $a_2 = 0$ in je zato $B\in V_{E_{11}}$. Sledi, da je $V_{E_{11}}$ strnjena. Ta premislek nam tudi pokaže, da je $V_{E_{11}}$ definirana kot zgoraj res maksimalna strnjena podgrupa, ki ne vsebuje $E_{11}$. Če bi namreč na mestu $(1,1)$ vsebovala nek neničelen element, bi morala ta podgrupa zaradi strnjenosti in  definicije urejenosti vsebovati tudi $E_{11}$.

Trdim, da je 
$$
V_{E_{12}} = \big\{
\begin{bmatrix}
0 & 0 \\
c_1 & d_1 \\
\end{bmatrix};
c_1 , d_1 \in \Z
\big\}.
$$
Jasno je to podgrupa, ki ne vsebuje $E_{12}$. Denimo, da za neko matriko $B$ oblike kot zgoraj velja, da je $|B| \le |A|, A \in V_{E_{12}}$. To je ekvivalentno temu, da je $|a_2| < 0$ ali $|a_2| = 0 $ in $|b_2| < 0$ ali $|a_2| = 0 $ in $|b_2| = 0 $ in $|c_2| \le |c_1|$ in $|d_2| \le |d_1|$. Spet vidimo, da je to lahko res le v tretjem primeru, torej je $a_2 = b_2 =0$ in je zato $B\in V_{E_{12}}$. Podgrupa je torej strnjena. Premislimo še, da je maksimalna strnjena podgrupa, ki ne vsebuje elementa $E_{12}$. Če bi vrednost elementa $E_{12}$ vsebovala nek element, ki ima na mestu $(1,1)$ nekaj neničelnega, bi zaradi strnjenosti vsebovala element $E_{12}$ (prva možnost iz definicije urejenosti), to pa ne gre. Podobno bi v primeru, da bi na mestu $(1,2)$ vsebovala neničelen element zaradi strnjenosti po drugi možnosti iz definicije urejenosti vsebovala element $E_{12}$, kar nas spet privede v protislovje z definicijo vrednosti. Sklenemo, da je zgoraj definirana podgrupa res vrednost elementa $E_{12}$.

Vrednost elementa $E_{21}$ je 
$$
V_{E_{21}} = \big\{
\begin{bmatrix}
0 & 0 \\
0 & d_1 \\
\end{bmatrix};
d_1 \in \Z
\big\}.
$$
To je res podgrupa, ki ne vsebuje elementa $E_{21}$. Preverimo, da je strnjena. Naj bo $B$ tak kot zgoraj in $|B| \le |A|, A\in V_{E_{21}}$. To pomeni da je $|a_2| < 0$ ali $ |a_2| =0 $ in $|b_2|<0$ ali $|a_2| = 0 $ in $|b_2| = 0$ in $|c_2|\le 0$ in $|d_2| \le | d_1|$. To implicira, da je $a_2 = b_ 2 = c_2 = 0$ in je zato $B\in V_{E_{21}}$, torej je to strnjena podgrupa. Premislimo še, da je maksimalna strnjena podgrupa, ki ne vsebuje elementa $E_{21}$. Podobno kot zgoraj, bi v primeru, da bi vrednost elementa $E_{21}$ vsebovala matriko, ki bi imela nekaj neničelnega na mestu $(1,1)$ ali $(1,2)$ zaradi strnjenosti takoj sledilo, da vsebuje še $E_{21}$, kar pa ne gre. Seveda tudi v primeru, ko bi na mestu $(2,1)$ vsebovala nekaj neničelnega (v prvi vrstici pa le dve ničli) po tretji možnosti iz definicije urejenosti zaradi strnjenosti podgrupe sledi, da vsebuje $E_{21}$, kar pa seveda ne gre. Našli smo torej vrednost elementa $E_{21}$.

Vrednost elementa $E_{22}$ pa je 
$$
V_{E_{22}} = \big\{
\begin{bmatrix}
0 & 0 \\
c_1 & 0 \\
\end{bmatrix};
c_1 \in \Z
\big\}. 
$$
To je res podgrupa, ki ne vsebuje $E_{22}$. Enako kot v prejšnjem primeru vidimo, da je strnjena. Na enak način kot v prejšnjem primeru vidimo še, da je to res maksimalna strnjena podgrupa, ki ne vsebuje elementa $E_{22}$, torej je to res vrednost $E_{22}$.

\item[(c)] Dokazati moramo, da smo v prejšnji točki dobili vse prapodgrupe grupe $G$. To preprosto sledi iz zgornjih premislekov o maksimalnosti strnjene podgrupe, ki ne vsebuje določenega elementa. Če imamo podgrupo, ki ima na mestu $(1,1)$ nekaj neničelnega, so zaradi strnjenosti in definicije urejenosti v njej tudi vse štiri matrične enote in dobimo kar celo grupo $G$. Denimo torej, da imamo podgrupo, ki vsebuje le matrike, ki imajo 0 na mestu $(1,1)$, drugje pa poljuben element. Tako dobimo $V_{E_{11}}$ (zaradi strnjenosti to podgrupo dobimo vedno, ko podgrupa vsebuje matriko, ki ima na mestu $(1,2)$ nekaj neničelnega) . Denimo sedaj, da imamo na mestih $(1,1),(1,2)$ v vseh matrikah iz podgrupe same ničle. Če imamo v podgrupi kako matriko, ki ima na mestu $(2,1)$ nekaj neničelnega in še neko matriko, ki ima na mestu $(2,2)$ neničelen element, potem zaradi strnjenosti dobimo podgrupo $V_{E_{12}}$. Če podgrupa vsebuje le matrike, ki imajo neničelne elemente le na mestu $(2,1)$ (oziroma $(2,2)$), potem zaradi strnjenosti vsebuje element $E_{21}$ (oziroma $E_{22}$) in je zato enaka $V_{E_{22}}$ (oziroma $V_{E_{21}}$). Če podgrupa ne vsebuje nobene matrike, ki bi imela na mestu $(2,1)$ ali $(2,2)$ nekaj neničelnega, potem ta podgrupa ne more vsebovati nobene matrične enote in je zato to trivialna podgrupa. 
\item[(d)]
\end{enumerate}

\begin{flushleft}
6. naloga
\end{flushleft}
Naj bo $(F,<)$ urejen obseg opremljen z intervalsko topologijo (to je topologija, ki je generirana z odprtimi intervali s krajišči v $F$). Dokažimo, da so naslednje preslikave zvezne: vsota, produkt (kot preslikavi iz $F\times F$ v $F$) in inverz (kot preslikava iz $F \setminus \{0\}$ v $F\setminus \{0\}$).

\begin{flushleft}
7. naloga
\end{flushleft}

\begin{flushleft}
8. naloga
\end{flushleft}

\begin{flushleft}
9. naloga
\end{flushleft}

\begin{flushleft}
10. naloga
\end{flushleft}

\end{document}