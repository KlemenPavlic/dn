% Začetek preambule
\documentclass[a4paper, 12pt]{article}
\usepackage[slovene]{babel}
\usepackage[utf8]{inputenc}
\usepackage[T1]{fontenc}
\usepackage{lmodern}
\usepackage{amsfonts,amsmath,amssymb}
\usepackage{graphicx}


% Moji ukazi, okolja,...
% oznake za števila
\newcommand{\N}{\mathbb{N}}
\newcommand{\Z}{\mathbb{Z}}
\newcommand{\Q}{\mathbb{Q}}
\newcommand{\R}{\mathbb{R}}
\newcommand{\C}{\mathbb{C}}
\newcommand{\F}{\mathbb{F}}
% opombe
\newenvironment{opomba}{\begin{flushleft} \textbf{Opomba}:}{\hfill \end{flushleft}}
% definicije
\newcounter{definitionCounter}
\addtocounter{definitionCounter}{1}
\newenvironment{definicija}{\begin{flushleft} \textit{\textbf{Definicija \arabic{definitionCounter}}}:}{\hfill \end{flushleft}\stepcounter{definitionCounter}}
% pojmi
\newcommand{\pojem}[1]{\textsc{#1}}
% dokazi
\newenvironment {dokaz}{\begin{flushleft} \textit{\textbf{Dokaz}}:}{\hfill $\square$\end{flushleft}}
% izreki
\newcounter{theoremCounter}
\addtocounter{theoremCounter}{1}
\newcounter{theoremCorollaryCounter}
\addtocounter{theoremCorollaryCounter}{0}
\newenvironment {izrek}{\begin{flushleft} \textsf{\textbf{IZREK \arabic{theoremCounter}}}:}{\hfill \end{flushleft}\stepcounter{theoremCounter}\stepcounter{theoremCorollaryCounter}\setcounter{corollaryCounter}{1}}
% leme
\newcounter{lemmaCounter}
\addtocounter{lemmaCounter}{1}
\newenvironment{lema}{\begin{flushleft} \textbf{Lema \arabic{lemmaCounter}}:}{\hfill \end{flushleft}\stepcounter{lemmaCounter}}
% posledice
\newcounter{corollaryCounter}
\addtocounter{corollaryCounter}{1}
\newenvironment  {posledica}{\begin{flushleft} \textsf{\textbf{Posledica \arabic{theoremCorollaryCounter}.\arabic{corollaryCounter}}}:}{\hfill \end{flushleft}\stepcounter{corollaryCounter}}
% dodatni ukazi
\usepackage{hyperref} % mora biti zadnji

% začetek dokumenta
\begin{document}
\thispagestyle{empty}
\noindent{\large
UNIVERZA V LJUBLJANI\\[1mm]
FAKULTETA ZA MATEMATIKO IN FIZIKO\\[5mm]
MATEMATIKA -- 2.~stopnja}
\vfill

\begin{center}{\large 
Klemen Pavlič\\[2mm]
{\bf Urejenostne algebrske strukture}\\[10mm]
Domača naloga}
\end{center}
\vfill

\noindent{\large
Velika Lašna, 2013}
\pagebreak
\newpage 

\begin{flushleft}
1. naloga
\end{flushleft}
Naj bo $G=\Z\times \Z$ z operacijo $(a,b)\circ (c,d) = (a+(-1)^b c, b+d)$.
\begin{enumerate}
\item[(a)] Dokažimo, da je $G$ grupa in da nima nobene linearne  ureditve.
\item[(b)] Dokažimo, da $G$ nima nobene mrežne ureditve.
\item[(c)] Konstruiraj na $G$ kako usmerjeno delno ureditev.
\end{enumerate}
\emph{Rešitev}
\begin{enumerate}
\item[(a)] Preverimo najprej, da je grupa. Jasno je, da je množica $\Z \times \Z$ zaprta za $\circ$. Preverimo asociativnost.
$$
\big[(a,b)\circ (c,d)\big] \circ (e,f) = (a+(-1)^b c, b+d) \circ (e,f) =
$$
$$
(a+(-1)^b c+ (-1)^{b+d}e, b+d+f),
$$
$$
(a,b)\circ \big[ (c,d) \circ (e,f) \big] = (a,b) \circ (c+(-1)^d e, d+f) =
$$
$$
(a+(-1)^b (c+(-1)^de), b+(d+f)) = (a+ (-1)^b c + (-1)^{b+d} e, b+d+f).
$$
Element $(0,0)$ je enota. Res, velja namreč
$$
(a,b)\circ (0,0) = (a+(-1)^b 0, b+0) = (a,b),
$$
$$
(0,0) \circ (a,b) = (0+(-1)^0 a, 0+b) = (a,b). 
$$
Inverz elementa $(a,b)$ je element $(a(-1)^{1-b},-b)$. Res, velja namreč
$$
(a,b) \circ (a(-1)^{1-b},-b) = (a+(-1)^b a (-1)^{1-b}, b-b) = (a-a,0 )= (0,0),
$$
$$
(a(-1)^{1-b}, -b) \circ (a,b) = (a(-1)^{1-b}+(-1)^{-b}a,-b+b)=
$$
$$
(a(-1)^{-b}(-1+1),0 ) = (0,0).
$$
Sledi, da je $(G,\circ)$ res grupa.

Pokažimo še, da $G$ nima nobene linearne ureditve. Denimo nasprotno, da na $G$ obstaja linearna ureditev. 

\item[(b)] 

\item[(c)] Definirajmo pozitiven stožec $P_{\ge} = \{(a,b); b > 0, a \text{ poljuben}\} \cup \{(0,0)\}$. Preverimo najprej, da je to res pozitiven stožec. Najprej preverimo, da velja $P_{\ge} \circ P_{\ge} \subseteq P_{ \ge}$. Naj bosta $(a,b), (c,d) \in P_{\ge}$. Če sta $b,d > 0$, potem je druga komponenta $(a,b)\circ (c,d)$ enaka $b+d > 0$ in je $(a,b)\circ (c,d) \in P_{\ge}$. Če je $d=0$, mora biti tudi $c=0$. Ker je $(0,0)$ enota, je $(a,b)\circ (0,0) = (a,b)$ in ima drugo komponento pozitivno, torej je v pozitivnem stožcu $P_{\ge}$. Enako velja za primer, ko $b=0$ in $d>0$. Če pa sta $b=d=0$, mora biti še $a=c=0$ in je $(a,b)\circ(c,d) = (0,0) \in P_{\ge}$. Sledi $P_{\ge} \circ P_{\ge} \subseteq P_{\ge}$.

Pokažimo, da je $P_{\ge} \cap P_{\ge} = \{(0,0)\}$. Naj bo element $(a,b)$ v preseku. Potem mora veljati $(a,b) \ge 0$ in $(a,b)^{-1}=(a(-1)^{1-b},-b)\ge 0$. To je res natanko tedaj, ko je bodisi $b>0$ in $-b>0$ bodisi $a=b=0$ in $a(-1)^{1-b}=-b=0$. V prvem primeru pridemo do protislovja, torej mora veljati drugi primer, to pa pomeni, da je $a=b=0$ in da je v preseku res le enota.

Naj bo zdaj še $(c,d)\in G$ poljuben element in $(a,b)\in P_{\ge}$ poljuben element. Preverimo, da je $(c,d)\circ (a,b) \circ (c,d)^{-1} \in P_{\ge}$. Velja
$$
(c,d)\circ (a,b) \circ (c,d)^{-1} = (c,d) \circ (a,b) \circ (c(-1)^{1-d},-d) =
$$
$$
(c+(-1)^d a ,d+b) \circ (c(-1)^{1-d},-d)=
$$
$$
(c+(-1)^d a +(-1)^{b+d}c(-1)^{1-d},d+b-d)=
$$
$$
(c+(-1)^d a + (-1)^{1+b}c,b).
$$
Če je $b>0$ , potem je ta element iz $P_{\ge}$. Če pa je $b=0$, je tudi $a=0$ in zgornje naprej poenostavimo v $(c+(-1)^d a + (-1)^{1+b}c,b) = (c+(-1)c,0)=(0,0) \in P_{\ge}$. Sklenemo, da je $P_{\ge}$ res pozitiven stožec.

Pokažimo sedaj, da je $G$ v tej ureditvi usmerjena. Naj bosta torej $(a,b), (c,d)\in G$ poljubna elementa. Poiskati moramo njuno spodnjo in zgornjo mejo. Trdim, da je element $(0,\max\{b,d\} + 1)$ zgornja meja. Preverimo, da je $(a,b)\le (0,\max\{b,d\} + 1)$. To bo res natanko tedaj, ko bo $(0,\max\{b,d\} + 1)\circ (a,b)^{-1} = (0,\max\{b,d\} +1)\circ (a(-1)^{1-b},-b)=((-1)^{\max\{b,d\}+1} a (-1)^{1-b},\max\{b,d\} - b +1 ) \in P_{\ge}$. To pa velja, ker je druga komponenta $\max\{b,d\} - b +1 \ge 1 > 0$. Podobno vidimo, da je zgornji element tudi zgornja meja za $(c,d)$.

Trdim, da je element $(0,\min\{b,d\} - 1)$ spodnja meja za $(a,b)$ in $(c,d)$. Preverimo, da je $(0,\min\{b,d\} - 1) \le (a,b)$. Podoben račun kot zgoraj pokaže, da je to res natanko tedaj, ko je $(a,b)\circ (0,\min\{b,d\} -1)^{-1} = (a,b+1-\min\{b,d\}) \in P_{\ge}$. To pa spet velja, ker je druga komponenta $b+1-\min\{b,d\} \ge 1 > 0$. Enako se pokaže še $(0,\min\{b,d\} - 1) \le  (c,d)$. Sledi, da je $G$ usmerjena.
\end{enumerate}

\begin{flushleft}
2. naloga
\end{flushleft}
\emph{Rešitev}

\begin{flushleft}
3. naloga
\end{flushleft}
\emph{Rešitev}

\begin{flushleft}
4. naloga
\end{flushleft}
\emph{Rešitev}

\begin{flushleft}
5. naloga
\end{flushleft}
Naj bo $G$ aditivna  grupa vseh celoštevilskih $2\times 2$ matrik z delno urejenostjo
$$
\begin{bmatrix}
a_1 & b_1 \\
c_1 & d_1 \\
\end{bmatrix}
\le 
\begin{bmatrix}
a_2 & b_2 \\
c_2 & d_2 \\
\end{bmatrix}
\Leftrightarrow
\begin{array}{l}
a_1 < a_2 \text{ ali}\\
a_1 = a_2 \text{ in } b_1 < b_2 \text{ ali}\\
a_1 = a_2 \text{ in } b_1 = b_2 \text{ in } c_1 \le c_2 \text{ in } d_1 \le d_2\\
\end{array}.
$$
\begin{enumerate}
\item[(a)] Dokaži, da je $G$ mrežno urejena.
\item[(b)] Izračunaj vrednosti koordinatnih matrik $E_{ij}$.
\item[(c)] Dokaži, da smo v (b) dobili vse prapodgrupe  v $G$.
\item[(d)] Opiši Hahnovo vložitev za grupo $G$.
\end{enumerate}
\emph{Rešitev}
\begin{enumerate}
\item[(a)]Trdim, da je množica
$$
P=\big\{
\begin{bmatrix}
a_1 & b_1 \\
c_1 & d_1 \\
\end{bmatrix}
;
\begin{array}{l}
a_1 >0 \text{ ali}\\
a_1 = 0 \text{ in } b_1 > 0 \text{ ali}\\
a_1 = 0 \text{ in } b_1 = 0 \text{ in } c_1 \ge 0 \text{ in } d_1 \ge 0\\
\end{array}
\big\}
$$
pozitiven stožec. Če ločimo nekaj primerov, z lahkoto pokažemo, da je $P+P \subseteq P$. Pokažimo še, da je $P\cap -P = \{0\}$. Če je 
$\begin{bmatrix}
a_1 & b_1 \\
c_1 & d_1\\
\end{bmatrix}
$
iz $P\cap -P$, potem mora veljati $a_1 > 0$ ali $(a_1 = 0 \text{ in } b_1 > 0)$ ali $(a_1=0 \text{ in } b_1 = 0 \text{ in } c_1 \ge 0 \text{ in } d_1 \ge 0)$ in hkrati še  $-a_1 > 0$ ali $(-a_1 = 0 \text{ in } -b_1 > 0)$ ali $(-a_1=0 \text{ in } -b_1 = 0 \text{ in } -c_1 \ge 0 \text{ in } -d_1 \ge 0)$. Sledi, da mora veljati $a_1 =b_1 =0$ in $0\le c_1 \le 0, 0 \le d_1 \le 0$, torej še $c_1 = d_1 = 0$. Sledi, da je v preseku res samo enota in ker je ta grupa komutativna, je ta množica res pozitiven stožec. 
Pokažimo še, da je mreža v tej urejenosti. Naj bosta 
$
A=
\begin{bmatrix}
a_1 & b_1 \\
c_1 & d_1 \\
\end{bmatrix}
$
in
$
B = 
\begin{bmatrix}
a_2 & b_2 \\
c_2 & d_2 \\
\end{bmatrix}
$
poljubni matriki. Poiščimo njun supremum. Če je $a_1 > a_2 $, potem je $A \lor B = A$ in podobno  je v primeru  $a_1 < a_2$ $A\lor B = B$. Naj bo torej zdaj $a_1 = a_2$. Če je še $ b_1 < b_2 $ je $A\lor B = B$, če pa je $b_1 > b_2$, pa je $A \lor B = A$. Opazujmo zdaj primer, ko je $a_1 = a_2 $ in $b_1 = b_2$. Ločimo še nekaj primerov. Če je 
\begin{itemize}
\item $c_1 \ge c_2 $ in $d_1\ge d_2$ je $A\lor B = A$,
\item $c_1 \le c_2$ in $d_1 \le d_2$ je $A\lor B = B$,
\item $c_1 \ge c_2$ in $d_2 \ge d_1$ je  $A\lor B=
\begin{bmatrix}
a_1 & b_1\\
c_1 & d_2 \\
\end{bmatrix}$,
\item $c_2 \ge c_1$ in $d_1 \ge d_2$ je  $A\lor B=
\begin{bmatrix}
a_1 & b_1\\
c_2 & d_1 \\
\end{bmatrix}$.
\end{itemize}
Vsi ti elementi so res pravi supremumi. Podobno poiščemo še infimume. Sledi, da je $G$ v dani urejenosti res mreža.
\item[(b)] 
\item[(c)]
\item[(d)]
\end{enumerate}

\end{document}