% Začetek preambule
\documentclass[a4paper, 12pt]{article}
\usepackage[slovene]{babel}
\usepackage[utf8]{inputenc}
\usepackage[T1]{fontenc}
\usepackage{lmodern}
\usepackage{amsfonts,amsmath,amssymb}
\usepackage{graphicx}


% Moji ukazi, okolja,...
% oznake za števila
\newcommand{\N}{\mathbb{N}}
\newcommand{\Z}{\mathbb{Z}}
\newcommand{\Q}{\mathbb{Q}}
\newcommand{\R}{\mathbb{R}}
\newcommand{\C}{\mathbb{C}}
\newcommand{\F}{\mathbb{F}}
% opombe
\newenvironment{opomba}{\begin{flushleft} \textbf{Opomba}:}{\hfill \end{flushleft}}
% definicije
\newcounter{definitionCounter}
\addtocounter{definitionCounter}{1}
\newenvironment{definicija}{\begin{flushleft} \textit{\textbf{Definicija \arabic{definitionCounter}}}:}{\hfill \end{flushleft}\stepcounter{definitionCounter}}
% pojmi
\newcommand{\pojem}[1]{\textsc{#1}}
% dokazi
\newenvironment {dokaz}{\begin{flushleft} \textit{\textbf{Dokaz}}:}{\hfill $\square$\end{flushleft}}
% izreki
\newcounter{theoremCounter}
\addtocounter{theoremCounter}{1}
\newcounter{theoremCorollaryCounter}
\addtocounter{theoremCorollaryCounter}{0}
\newenvironment {izrek}{\begin{flushleft} \textsf{\textbf{IZREK \arabic{theoremCounter}}}:}{\hfill \end{flushleft}\stepcounter{theoremCounter}\stepcounter{theoremCorollaryCounter}\setcounter{corollaryCounter}{1}}
% leme
\newcounter{lemmaCounter}
\addtocounter{lemmaCounter}{1}
\newenvironment{lema}{\begin{flushleft} \textbf{Lema \arabic{lemmaCounter}}:}{\hfill \end{flushleft}\stepcounter{lemmaCounter}}
% posledice
\newcounter{corollaryCounter}
\addtocounter{corollaryCounter}{1}
\newenvironment  {posledica}{\begin{flushleft} \textsf{\textbf{Posledica \arabic{theoremCorollaryCounter}.\arabic{corollaryCounter}}}:}{\hfill \end{flushleft}\stepcounter{corollaryCounter}}
% dodatni ukazi
\usepackage{hyperref} % mora biti zadnji

% začetek dokumenta
\begin{document}
\thispagestyle{empty}
\noindent{\large
UNIVERZA V LJUBLJANI\\[1mm]
FAKULTETA ZA MATEMATIKO IN FIZIKO\\[5mm]
MATEMATIKA -- 2.~stopnja}
\vfill

\begin{center}{\large 
Klemen Pavlič\\[2mm]
{\bf Urejenostne algebrske strukture}\\[10mm]
Domača naloga}
\end{center}
\vfill

\noindent{\large
Velika Lašna, 2013}
\pagebreak
\newpage 

\begin{flushleft}
1. naloga
\end{flushleft}
Naj bo $G=\Z\times \Z$ z operacijo $(a,b)\circ (c,d) = (a+(-1)^b c, b+d)$.
\begin{enumerate}
\item[(a)] Dokažimo, da je $G$ grupa in da nima nobene linearne  ureditve.
\item[(b)] Dokažimo, da $G$ nima nobene mrežne ureditve.
\item[(c)] Konstruiraj na $G$ kako usmerjeno delno ureditev.
\end{enumerate}
\emph{Rešitev}
\begin{enumerate}
\item[(a)] Preverimo najprej, da je grupa. Jasno je, da je množica $\Z \times \Z$ zaprta za $\circ$. Preverimo asociativnost.
$$
\big[(a,b)\circ (c,d)\big] \circ (e,f) = (a+(-1)^b c, b+d) \circ (e,f) =
$$
$$
(a+(-1)^b c+ (-1)^{b+d}e, b+d+f),
$$
$$
(a,b)\circ \big[ (c,d) \circ (e,f) \big] = (a,b) \circ (c+(-1)^d e, d+f) =
$$
$$
(a+(-1)^b (c+(-1)^de), b+(d+f)) = (a+ (-1)^b c + (-1)^{b+d} e, b+d+f).
$$
Element $(0,0)$ je enota. Res, velja namreč
$$
(a,b)\circ (0,0) = (a+(-1)^b 0, b+0) = (a,b),
$$
$$
(0,0) \circ (a,b) = (0+(-1)^0 a, 0+b) = (a,b). 
$$
Inverz elementa $(a,b)$ je element $(a(-1)^{1-b},-b)$. Res, velja namreč
$$
(a,b) \circ (a(-1)^{1-b},-b) = (a+(-1)^b a (-1)^{1-b}, b-b) = (a-a,0 )= (0,0),
$$
$$
(a(-1)^{1-b}, -b) \circ (a,b) = (a(-1)^{1-b}+(-1)^{-b}a,-b+b)=
$$
$$
(a(-1)^{-b}(-1+1),0 ) = (0,0).
$$
Sledi, da je $(G,\circ)$ res grupa.
\item[(b)] 
\item[(c)] 
\end{enumerate}

\begin{flushleft}
2. naloga
\end{flushleft}

\begin{flushleft}
3. naloga
\end{flushleft}

\begin{flushleft}
4. naloga
\end{flushleft}

\begin{flushleft}
5. naloga
\end{flushleft}


\end{document}